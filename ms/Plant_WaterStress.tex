\documentclass[a4paper,11pt]{article}
\usepackage[osf]{mathpazo}
\usepackage{ms}
\usepackage{natbib}
\usepackage{graphicx}
\usepackage{caption}
\usepackage{hyperref}
%% \usepackage{authblk}
\usepackage[labelfont=bf]{caption} % make label for figure bold

% Allow referencing into the supporting information, once that exists.
\IfFileExists{./Plant_WaterStress_SupplMat.tex}{%
  \usepackage{xr}%
  \externaldocument{Plant_WaterStress_SupplMat}}{}

% We will generate all images so they have a width \maxwidth. This means
% that they will get their normal width if they fit onto the page, but
% are scaled down if they would overflow the MAPgins.
\makeatletter
\def\maxwidth{\ifdim\Gin@nat@width>\linewidth\linewidth\else\Gin@nat@width\fi}
\def\maxheight{\ifdim\Gin@nat@height>\textheight\textheight\else\Gin@nat@height\fi}
\makeatother
\setkeys{Gin}{width=\maxwidth,height=\maxheight,keepaspectratio}

\title{Change of LMA and Leaf N per area along a water stress gradient}

\author{Georges Kunstler, Daniel S. Falster, Richard G. FitzJohn}
\date{}
\affiliation{Irstea, Grenoble, France and Department of Biological Sciences, Macquarie University,
  Sydney, Australia}
\date{}
\runninghead{}
\keywords{}

\usepackage{color}

\newcommand{\ud}{\ensuremath{\mathrm{d}}}
\newcommand{\sign}{\mathop{\mathrm{sign}}\nolimits}
\newcommand{\Rstar}{\ensuremath{R^*}}
\newcommand{\plant}{{\tt plant}}
\newcommand{\hmat}{\ensuremath{h_{\text{mat}}}}
\newcommand{\TODO}{{\color{red}\sc todo}}


\begin{document}

\mstitleshort
%% \mstitlepage
\parindent=1.5em
\addtolength{\parskip}{.3em}

% \begin{abstract}
% Abstract goes here\ldots
% \end{abstract}

\section{Background \& Motivation}

Understanding the divers of changes in vegetation composition across large biogeographic gradients is fundamental because (i) vegetation is key to control ecosystem functioning, such as net primary production, and carbon storage, (ii) spatial turnover in species composition explains a large proportion of the global diversity ($\beta$ diversity), and (iii) understanding these drivers is a path towards predicting how vegetation may change as result of environmental change.

% NEED TO BETTER LINK HERE TRAITS HAVE BEEN PROPOSED TO UNRAVEL THESE DRIVERS
Beyond species succession pattern along climatic gradients, several
studies document gradient of functional traits along climatic
gradients. Because functional traits are directly related to plant functioning these traits gradients raised the hope to better understand drivers of changes in vegetation. One specificity of observed pattern of traits variation is
that they are generally weak with generally a higher variation within
sites than along the climatic gradients \citep[see][]{Wright-2004}. A
classical example is the variation of leaf mass per area ($LMA$) and
leaf nitrogen per area ($N_{area}$) with aridity. $LMA$
\citep{Wright-2004,Onoda-2011,Moles-2014} and $N_{area}$
\citep{Wright-2005,Maire-2015} both increase with aridity, but these
patterns are very weak.


Plot of pattern for leaf traits extracted from glopnet \citep{Wright-2004}.

\begin{figure}[ht]
\centering
\includegraphics{../figures/leaf_climate.pdf}
\caption{\textbf{Pattern in leaf mass per area and leaf nitrogen per area along water availability index (a and d) and MAT over MAP (b and e), and MAP over PET (c and f).} Data are from glopnet \citep{Wright-2004} and MAP over PET is from \url{http://www.cgiar-csi.org/data/global-aridity-and-pet-database}.
\label{fig:leafpattern}}
\end{figure}

\clearpage

\section{Potential mechanisms for these patterns}

Documenting empirical patterns of variation of plant traits along
climatic gradient represents a key first step to understand changes in
vegetation composition, but we also need to propose mechanisms to
explain these traits patterns. 


There is a long history of studies
proposing adaptative arguments based on optimisation approaches with
ecophysiological models of plant functioning \citep{Makela-2002}. For instance, high $LMA$ has been proposed as an advantage in dry climate because
stiffer leaves can have a higher resistance to dehydration
\citep{Wright-2002a,Wright-2002b} or the need to avoid overheating
under condition of low transpiration \citep{Leigh-2012}.

%% Higher
%% $N_{area}$ in dry climate have also been proposed as strategy to
%% maximise photosynthesis \citep{Wright-2003}. This is
%% based on the idea that plant are minimising the summed cost of water
%% loss and carbon gain
%% \citep{Medlyn-2002,Wright-2003,Prentice-2014,Lu-2016,Wang-2017,Dong-2017}. As
%% higher $N_{area}$ translate into higher $V_{cmax}$ this allow the
%% plant to achieve a high $A_{area}$ even at the low stomatal conductance
%% $g_s$ encountered in dry climate.

Optimisation approaches provide, however, no explanations to the large range of trait variation observed
within each site. This is because they fail to account for the fact that
these traits also controls species interaction and the coexistence of
species with different traits values. 
Theoretical models \citep{Case-2000,Goldberg-2006,Leimar-2008}
studies have shown that abiotic constrains and competitive interactions jointly control species coexistence \citep{Chesson-2018} and the range of traits
value found in a given community and along environmental gradients. Competition can clearly make the traits deviate from the climatic optimum.  Generally, theses theoretical models are, however, only weakly connected with the physiological processes, real traits, and abiotic gradients and provides no prediction to be compared with field data. 

Recently a few studies have started to connect physiological approach
and the theoretical approach to build trait-based models of full
communities dynamics either at local \citep{Farrior-2013} or global
scale \citep[see][]{Sakschewski-2015,Scheiter-2013}. They represent key elements of plant ecophysiological functioning. But they have been largely unsuccessful at explaining the wide range of traits found in a single community as these models generally failed to capture the processes promoting species coexistence.

It is thus key to develop mechanistic models where traits are related
to both the direct climate effect and to species coexistence. We propose to do this with a new model -- \plant\
\citep{Falster-2016,Falster-2017} -- that connects directly with leaf
traits and simultaneously allows for coexistence within sites and
variation across sites. Coexistence is driven by competition for light
in size-structured meta-populations and the outcomes of competition is
moderated by environmental factors such as site productivity and disturbance regime.

Here we propose to focus on two traits, $LMA$ and $N_{area}$ which are know to be related to competition for light and water stress gradients. These two traits are among the most common traits and we start to have a deep understanding of their effect on pant functioning. 

\subsection{$LMA$ and $N_{area}$ response to light}
 In \plant\ the coexistence between tree is mainly related to difference in successional niche underpinned by $LMA$ through competition for light. The coordination between $LMA$ and $LSS$ discriminates different strategies of resource use. Species with low $LMA$ have a greater potential maximum growth and are thus favoured in high light condition, whereas high $LMA$ species have longer leaf life span and thus are more conservative of their resource and are favoured in low light condition. Low $N_{area}$ is related to a higher shade-tolerance for small trees \citep{Falster-2018}, because of a higher respiration cost.


 
\subsection{$LMA$ and $N_{area}$ response to aridity}

High $LMA$ has been proposed as an advantage in dry climate because
stiffer leaves can have a higher resistance to dehydration
\citep{Wright-2002a,Wright-2002b} or the need to avoid overheating
under condition of low transpiration \citep{Leigh-2012}.
Pattern of variation of $LMA$ along aridity gradient is extremely weak and it is unclear whether this is directly related to an adaptive value to the abiotic stress or to a response to a change in productivity with aridity.
 
$N_{area}$ is related to water stress tolerance and shade tolerance. Optimisation literature of $N_{area}$ predict that $N_{area}$ should
increase with aridity \citep{Wright-2003}. The general approach is
based on the idea that plant are minimising the summed cost of water
loss and carbon gain
\citep{Medlyn-2002,Wright-2003,Prentice-2014,Lu-2016,Wang-2017,Dong-2017}. As
higher $N_{area}$ translate into higher $V_{cmax}$ this allows the
plant to achieve a high $A_{area}$ even at low stomatal conductance
$g_s$. Recently \citet{Prentice-2014} and \citet{Dong-2017} provided
quantitative predictions for the effect of key climatic variable
(including VPD) based on an optimised Farquhar photosynthesis
model, showing how high leaf nitrogen could be selected in water stressed areas. Several authors have challenged the idea of summing the cost of
water and carbon gain on the ground that competition should favour
carbon gain optimisation only \citep{Wolf-2016} and the idea that the
main effect of water stress come through cavitation and mortality \citep{Sperry-2016,Sperry-2017}. Our knowledge on these alternative mechanisms is however too lacunar to be explored in \plant\ for the moment.

An additional effect of $N_{area}$ along aridity gradient is that higher $N_{area}$ could be related to a lower mechanical resistance of leaf resulting in a shift in the LMA \textit{vs.} leaf lifespan tradeoff to shorter lifespan \citep{Wright-2002a}.

%% \citep{Dong-2017} proposed that
%% $N_{area}$ could be divided in structural and rubisco related N and that this is structura(using
%% the relationship of \citet{Onoda-2004}) but this distinction of N in to group is beyond the current implementation of \plant\ .


\section{Key questions}

We propose to extend Plant to include a direct effect of water stress leaf photosynthesis by using a coupled Farquhar photosynthesis model and stomatal conductance model. This allows us to explore water stress effect on photosynthesis through the most simple and well know effect, a direct effect of vapour pressure deficit 'vpd' on leaf photosynthesis. This approach
  is based on well know mechanisms implemented in most DGVMs. This approach however represents only water stress effect through vpd and ignore water stress through soil water availability. This extended model allow us to explore a classical mechanisms which propose that higher $N_{area}$ could results in higher drought tolerance because higher $N_{area}$ is related to higher $V_{cmax}$ that allows the
plant to achieve a high $A_{area}$ even at low stomatal conductance
$g_s$ \citep{Wright-2003}. This $N_{area}$ advantage of could comes at an additional cost than just increase respiration by decreasing leaf lifespan because of lower mechanical resistance \citep{Wright-2002a}.  

Using this model how the assemblage of LMA and $N_{area}$ change along a gradient of drought stress represented by vpd. Metacommunities are assembled evolutionarily building on the adaptive dynamics framework. More specifically we test the following hypothesis:  
 
\begin{itemize}

\item Light competition selects for a negative correlation between $LMA$ and $N_{area}$ as shade tolerance is related to high $LMA$ and low $N_{area}$ (and the inverse for high light staregies).

\item Higher vpd shift community assembly toward higher $N_{area}$ and smaller range of traits.

\item Higher vpd shift community assembly toward higher $LMA$ but only slightly i comparison to the range of $LMA$ coexisting in a given meta-community.

\item Accounting for $N_{area}$ effect on leaf turnover rate had an additional cost of high $N_{area}$ due to faster leafturnover rate. This should counter select $N_{area}$ in low water stress area even for high light strategies but also affect $LMA$.

\end{itemize}


\section{Materials and Methods}


\begin{itemize}

\item Presentation of Plant

\item Adaptation of Plant to use a Farquhar photosynthesis model coupled with a stomatal conductance model

\item Trait based parameterisation of photosynthesis model

\item vpd variation at constant temperature
  
\item Simulations of community assembly 

\end{itemize}



\clearpage

\subsection{$LMA$ and $N_{area}$ correlation}

\begin{figure}[ht]
\centering
\includegraphics{../figures/gradient_narea_lma_single_FvCB_cor.pdf}
\caption{\textbf{LMA vs Leaf N area for FVCB model with high disturbance rate (Disturbance mean interval 5 years). The color of the points represents the vpd.}
\label{fig:lma_narea_cor_single}}
\end{figure}


\begin{figure}[ht]
\centering
\includegraphics{../figures/gradient_narea_lma_multi_FvCB_cor.pdf}
\caption{\textbf{LMA vs Leaf N area for FVCB model with low disturbance rate (Disturbance mean interval 40 years). The color of the points represents the vpd.}
\label{fig:lma_narea_cor_multi}}
\end{figure}


\begin{figure}[ht]
\centering
\includegraphics{../figures/gradient_narea_lma_single_NvLTR_FvCB_cor.pdf}
\caption{\textbf{LMA vs Leaf N area for model with link between N area and leaf turnover rate with high disturbance rate (Disturbance mean interval 5 years). The color of the points represents the vpd.}
\label{fig:lma_narea_cor_single_NvLTR}}
\end{figure}

\clearpage

\subsection{$LMA$ and $N_{area}$ along a vpd gradient}

\begin{figure}[ht]
\centering
\includegraphics{../figures/gradient_narea_lma_single_narea_lma_FvCB.pdf}
\caption{\textbf{Predicted community assembly of leaf mass per area and leaf N per area along a productivity gradient controlled by aridity by 'Plant' based on Farquhar with high disturbance rate (Disturbance mean interval 5). The size of the point represent the variation of LMA.}
\label{fig:lma_nareaFh}}
\end{figure}

\begin{figure}[ht]
\centering
\includegraphics{../figures/gradient_narea_lma_single_narea_lma2_FvCB.pdf}
\caption{\textbf{Predicted community assembly of leaf mass per area and leaf N per area along a productivity gradient controlled by aridity by 'Plant' based on Farquhar with high disturbance rate (Disturbance mean interval 5). The size of the point represent the variation of LMA.}
\label{fig:lma_nareaFh2}}
\end{figure}


\begin{figure}[ht]
\centering
\includegraphics{../figures/gradient_narea_lma_multi_narea_lma_FvCB.pdf}
\caption{\textbf{Predicted community assembly of leaf mass per area and leaf N per area along a productivity gradient controlled by aridity by 'Plant' based on Farquhar with low disturbance rate (Disturbance mean interval 40). The size of the point represent the variation of LMA.}
\label{fig:lma_nareaFl}}
\end{figure}

\begin{figure}[ht]
\centering
\includegraphics{../figures/gradient_narea_lma_multi_narea_lma2_FvCB.pdf}
\caption{\textbf{Predicted community assembly of leaf mass per area and leaf N per area along a productivity gradient controlled by aridity by 'Plant' based on Farquhar with low disturbance rate (Disturbance mean interval 40). The size of the point represent the variation of LMA.}
\label{fig:lma_nareaFl2}}
\end{figure}


\clearpage

\subsection{$LMA$ and $N_{area}$ along a vpd gradient with N area impact on leaf turnover rate}


\begin{figure}[ht]
\centering
\includegraphics{../figures/gradient_narea_lma_single_narea_lma_NvLTR_FvCB.pdf}
\caption{\textbf{Predicted community assembly of leaf mass per area and leaf N per area along a productivity gradient controlled by aridity by 'Plant' based on Farquhar and a link between N area and leaf turnover rate with high disturbance rate (Disturbance mean interval 5). The size of the point represent the variation of LMA.}
\label{fig:lma_nareaFh}}
\end{figure}

\begin{figure}[ht]
\centering
\includegraphics{../figures/gradient_narea_lma_single_narea_lma2_NvLTR_FvCB.pdf}
\caption{\textbf{Predicted community assembly of leaf mass per area and leaf N per area along a productivity gradient controlled by aridity by 'Plant' based on Farquhar and a link between N area and leaf turnover rate with high disturbance rate (Disturbance mean interval 5). The size of the point represent the variation of LMA.}
\label{fig:lma_nareaFh2}}
\end{figure}



\clearpage

\subsection{Discussion}



\begin{itemize}

\item explain range

\item interactions between traits in assembly

\item strong effect of N area - LTR model

\end{itemize}




\bibliographystyle{amnat}
\bibliography{references}
\end{document}
