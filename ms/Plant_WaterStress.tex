\documentclass[a4paper,11pt]{article}
\usepackage[osf]{mathpazo}
\usepackage{ms}
\usepackage{natbib}
\usepackage{graphicx}
\usepackage{caption}
\usepackage{hyperref}
%% \usepackage{authblk}
\usepackage[labelfont=bf]{caption} % make label for figure bold

% Allow referencing into the supporting information, once that exists.
\IfFileExists{./Plant_WaterStress_SupplMat_1.tex}{%
  \usepackage{xr}%
  \externaldocument{Plant_WaterStress_SupplMat_1}
  \externaldocument{Plant_WaterStress_SupplMat_2}
}{}

% We will generate all images so they have a width \maxwidth. This means
% that they will get their normal width if they fit onto the page, but
% are scaled down if they would overflow the MAPgins.
\makeatletter
\def\maxwidth{\ifdim\Gin@nat@width>\linewidth\linewidth\else\Gin@nat@width\fi}
\def\maxheight{\ifdim\Gin@nat@height>\textheight\textheight\else\Gin@nat@height\fi}
\makeatother
\setkeys{Gin}{width=\maxwidth,height=\maxheight,keepaspectratio}

\title{Change of LMA and Leaf N per area along a water stress gradient}

\author{Georges Kunstler, Daniel S. Falster, Richard G. FitzJohn}
\date{}
\affiliation{INRAE, Grenoble, France and Department of Biological Sciences, Macquarie University,
  Sydney, Australia}
\date{}
\runninghead{}
\keywords{}

\usepackage{color}

\newcommand{\ud}{\ensuremath{\mathrm{d}}}
\newcommand{\sign}{\mathop{\mathrm{sign}}\nolimits}
\newcommand{\Rstar}{\ensuremath{R^*}}
\newcommand{\plant}{{\tt plant}}
\newcommand{\hmat}{\ensuremath{h_{\text{mat}}}}
\newcommand{\TODO}{{\color{red}\sc todo}}


\begin{document}

\mstitleshort
%% \mstitlepage
\parindent=1.5em
\addtolength{\parskip}{.3em}

% \begin{abstract}
% Abstract goes here\ldots
% \end{abstract}

\section{Introduction}

Understanding the divers of changes in vegetation composition across large biogeographic gradients is fundamental because vegetation controls key ecosystem functions such as net primary production, and carbon storage and shelters a significant proportion of the global biodiversity. Understanding these drivers is a path towards predicting how vegetation may change as result of environmental change. Several studies document gradient of functional traits along climatic
gradients. Because functional traits are directly related to plant
functioning these traits gradients raised the hope to help better
understand drivers of changes in vegetation.

There is a long history of studies
proposing adaptative arguments based on optimisation approaches with
ecophysiological models of plant functioning \citep{Makela-2002} to
identify the drivers of the variation of traits along environmental
gradients.
% For instance, high $LMA$ and high $N_{area}$ has been proposed as an advantage in dry climate because
% a higher resistance to dehydration
% \citep{Wright-2002a,Wright-2002b}, lower risk of overheating
% under condition of low transpiration \citep{Leigh-2012}, and better photosynthesis performance at low stomatal conductance \citep{Wright-2003}.
Optimisation approaches provide, however, no explanations for the large range of trait variation observed
within each site which is however extremely large. A specificity of observed pattern of traits variation is
that they are generally weak with generally a higher variation within
sites than along the climatic gradients \citep[see][]{Wright-2004}. A
classical example is the variation of leaf mass per area ($LMA$) and
leaf nitrogen per area ($N_{area}$) with aridity. $LMA$
\citep{Wright-2004,Onoda-2011,Moles-2014} and $N_{area}$
\citep{Wright-2005,Maire-2015} both increase with aridity, but these
patterns are very weak, with more variability of the traits within
sites than between sites (see plots of pattern for leaf traits extracted from glopnet \citep{Wright-2004} in Figure. \ref{fig:leafpattern}).

\begin{figure}[ht]
\centering
\includegraphics{../figures/leaf_climate.pdf}
\caption{\textbf{Pattern in leaf mass per area and leaf nitrogen per area along water availability index (a and d) and MAT over MAP (b and e), and MAP over PET (c and f).} Data are from glopnet \citep{Wright-2004} and MAP over PET is from \url{http://www.cgiar-csi.org/data/global-aridity-and-pet-database}.
\label{fig:leafpattern}}
\end{figure}


The key limitation of the optimisation approach is that it fails to account for the fact that
traits also control species interaction and the coexistence of
species with different traits values \citep{Chesson-2018}. 
Abstract theoretical models \citep{Case-2000,Goldberg-2006,Leimar-2008} have shown that abiotic constrains and competitive interactions jointly control species coexistence and the range of traits
value found along environmental gradients and that competition can
clearly make the traits deviate from the climatic optimum. These
abstract theoretical models have however generally been disconnected
of the physiological mechanisms underpinning trait effects. Recently studies have started to extand the physiological optimisation
approach to include competitive interaction between plants to build trait-based models at local \citep{Farrior-2013} and global
scale \citep[see][]{Sakschewski-2015,Scheiter-2013}. But these models
have been largely unsuccessful at explaining the wide range of traits
found in a single community because they failed to capture the main
processes promoting species coexistence in plant communities.

In plant communities numerous coexistence mechanisms can be at
play. One of the most classical one for perennial plant is the
differentiation of successional strategies that may promote
coexistence, provided disturbances create a mosaic of successional
stages \citep{Falster-2017}. Successional strategy is thought to be underpinned by a tradeoff
between performance in high light and performance in low light (REF). Light
competition is included in several of the above mentionned
physiological models, but most of them do not represent the
meta-community structure and disturbance regime underpinning coexistence.

It is thus key to develop mechanistic models where traits are related
to both climate constraints and coexistence mechanisms differentiation of successional strategies. We propose to do this with a new model -- \plant\
\citep{Falster-2016,Falster-2017} -- that connects directly with leaf
traits and simultaneously allows for coexistence within sites. Coexistence is driven by competition for light
in size-structured meta-populations under a disturbance regime. Here
we propose to extand the existing model to represent community
assembly along a drought gradient focusing on two traits, $LMA$ and
$N_{area}$. We propose to focus on these two traits because: 1. they
 are know to be related to response to water stress and light
competition (REF), 2.  they are among the most commonly measured
traits in field studies, and 3. we start to have a deep understanding
of the mechanisms underpinning their effects on plant functioning. 

%% \subsection{$LMA$ and $N_{area}$ response to light}
Both $LMA$ and $N_{area}$ are related to light competition. The coordination between $LMA$ and $LSS$ discriminates different
strategies of light use \citep{Falster-2018}. Species with low $LMA$
have a greater potential maximum growth and are thus favoured in high
light condition, whereas high $LMA$ species have longer leaf life span
and thus are more conservative of their resource and are favoured in
low light condition \citep{Falster-2018}. High $N_{area}$ results in faster growth rate in
high light, but this advantage comes at the cost of higher respiration
rate resulting in poor performance in low light
\citep{Falster-2018}. Thus a high shade-tolerance might be related to
a high $LMA$ but a low $N_{area}$.


These two traits have also been proposed to be connected to drought
stress. High $LMA$ has been proposed as an advantage in dry climate
because high $LMA$ is generally selected in unproductive environment
caracteristic of dry climate. In addition, stiffer leaves can have a higher resistance to dehydration
\citep{Wright-2002a,Wright-2002b} or overheating
under condition of low transpiration \citep{Leigh-2012} but these
mechanisms are less well understood.
Pattern of variation of $LMA$ along aridity gradient is extremely weak and it is unclear whether this is directly related to an adaptive value to the abiotic stress or to a response to a change in productivity with aridity.
 
$N_{area}$ is related to water stress tolerance. Optimisation literature of $N_{area}$ predict that $N_{area}$ should
increase with aridity \citep{Wright-2003}, based on the idea that plant are minimising the summed cost of water
loss and carbon gain
\citep{Medlyn-2002,Wright-2003,Prentice-2014,Lu-2016,Wang-2017,Dong-2017}. As
higher $N_{area}$ translate into higher $V_{cmax}$ this allows the
plant to achieve a high $A_{area}$ even at low stomatal conductance
$g_s$.

So tolerance to drought might be related to high traits value of both
$LMA$ and $N_{area}$, leading to a selection in the same direction
along the drought gradient, but a counter selection by the light
competition. A side effect of high $N_{area}$
increase is that this could results in lower mechanical
resistance of leaf shifting the LMA \textit{vs.} leaf
lifespan tradeoff intercept toward shorter lifespan
\citep{Wright-2002a}. This additional effect might complexify the
selection on both traits along aridity graidents.


\section{Key questions}

To explore how the evolutionary
assemblage of $LMA$ and $N_{area}$ in successional meta-communities
driven by light competition changes along a water stress gradient we
extended the \plant\ model. We included a direct effect of water
stress on leaf photosynthesis in \plant\ by using a coupled Farquhar
photosynthesis model with a stomatal conductance model. Water stress
affect photosynthesis through an increase of vapour pressure deficit
'vpd'. We include an effect of $N_{area}$ on $V_{cmax}$ that can
translate into a higher drought tolerance as proposed in the
optimisation literature. This approach
is based on well know mechanisms implemented in most DGVMs (REF)
and represents a simple and well know effect of water stress on plant
functioning.

vpd vary over short term depending on short term
temperature changes and ignore soil water availability
limitations. However, over the long term and large spatial scale
variation in long-term mean vpd are expected to follows trends in the
availability of water for evaporation and transpiration
\citep{Prentice-2014}. The mechanisms of water stress are much more complex than a simple vpd
effect, as water stress also comes through soil water limitation leading to
both carbon gain limitation and cavitation and mortality
\citep{Sperry-2016,Wolf-2016,Sperry-2017}. But our understanding of
the trait basis to represent these mechanisms and how they translate
into competitive interactions is much more limited. We thus focus only
on vpd effect in the analysis and we will explore the role of soil water stress and water competition in futur version of the model.

With this extended version of \plant\ we build on the adaptive dynamics framework to simulate evolutionary assemblage of $LMA$ and $N_{area}$ in meta-communities at different mean vpd to test the following hypothesis:  
 
\begin{itemize}

\item Light competition selects for a negative correlation between $LMA$ and $N_{area}$ as shade tolerance is related to high $LMA$ and low $N_{area}$ (and the inverse for high light strategies).

\item Higher vpd shift community assembly toward higher $N_{area}$ and smaller range of traits.

\item Higher vpd shift community assembly toward higher $LMA$ but only slightly in comparison to the range of $LMA$ coexisting in a given meta-community.

\item Accounting for $N_{area}$ effect on leaf turnover rate adds an
  additional cost of high $N_{area}$ due to faster leaf turnover
  rate. This should counter select high $N_{area}$ in low water stress
  area even for high light strategies but also affect the value of $LMA$ selected.

\end{itemize}


\section{Materials and Methods}

TODO

\begin{itemize}

\item Presentation of Plant

\item Adaptation of Plant to use a Farquhar photosynthesis model coupled with a stomatal conductance model

\item Trait based parameterisation of photosynthesis model

\item vpd variation at constant temperature
  
\item Simulations of community assembly 

\end{itemize}



\clearpage

\section{Results}


\subsection{$LMA$ along a vpd gradient}


\subsubsection{$LMA$  shift with vpd}

\begin{figure}[ht]
\centering
\includegraphics{../figures/gradient_lma_single_lma_FvCB.pdf}
\caption{\textbf{Predicted community assembly of leaf mass per area along a productivity gradient controlled by aridity by 'Plant' based on Farquhar with high disturbance rate (Disturbance mean interval 5). The size of the point represent the variation of LMA.}
\label{fig:lma_vpd}}
\end{figure}

\clearpage

\subsection{$LMA$ and $N_{area}$ along a vpd gradient}
\subsubsection{$LMA$ and $N_{area}$ correlation}

\begin{figure}[ht]
\centering
\includegraphics{../figures/gradient_narea_lma_single_FvCB_cor.pdf}
\caption{\textbf{LMA vs Leaf N area for FVCB model with high disturbance rate (Disturbance mean interval 5 years). The color of the points represents the vpd.}
\label{fig:lma_narea_cor_single}}
\end{figure}


\clearpage

\subsubsection{$LMA$ and $N_{area}$ shift with vpd}


% \begin{figure}[ht]
% \centering
% \includegraphics{../figures/gradient_narea_lma_single_narea_lma_FvCB.pdf}
% \caption{\textbf{Predicted community assembly of leaf mass per area and leaf N per area along a productivity gradient controlled by aridity by 'Plant' based on Farquhar with high disturbance rate (Disturbance mean interval 5). The size of the point represent the variation of LMA.}
% \label{fig:lma_nareaFh}}
% \end{figure}

\begin{figure}[ht]
\centering
\includegraphics{../figures/gradient_narea_lma_single_narea_lma2_FvCB.pdf}
\caption{\textbf{Predicted community assembly of leaf mass per area and leaf N per area along a productivity gradient controlled by aridity by 'Plant' based on Farquhar with high disturbance rate (Disturbance mean interval 5). The size of the point represent the variation of LMA.}
\label{fig:lma_nareaFh2}}
\end{figure}




\clearpage

\subsection{$LMA$ and $N_{area}$ along a vpd gradient with N area impact on leaf turnover rate}

\subsubsection{$LMA$ and $N_{area}$ correlation}

\begin{figure}[ht]
\centering
\includegraphics{../figures/gradient_narea_lma_single_NvLTR_FvCB_cor.pdf}
\caption{\textbf{LMA vs Leaf N area for model with link between N area and leaf turnover rate with high disturbance rate (Disturbance mean interval 5 years). The color of the points represents the vpd.}
\label{fig:lma_narea_cor_single_NvLTR}}
\end{figure}

\clearpage

\subsubsection{$LMA$ and $N_{area}$ shift with vpd}

% \begin{figure}[ht]
% \centering
% \includegraphics{../figures/gradient_narea_lma_single_narea_lma_NvLTR_FvCB.pdf}
% \caption{\textbf{Predicted community assembly of leaf mass per area and leaf N per area along a productivity gradient controlled by aridity by 'Plant' based on Farquhar and a link between N area and leaf turnover rate with high disturbance rate (Disturbance mean interval 5). The size of the point represent the variation of LMA.}
% \label{fig:lma_nareaFh}}
% \end{figure}

\begin{figure}[ht] 
\centering
\includegraphics{../figures/gradient_narea_lma_single_narea_lma2_NvLTR_FvCB.pdf}
\caption{\textbf{Predicted community assembly of leaf mass per area and leaf N per area along a productivity gradient controlled by aridity by 'Plant' based on Farquhar and a link between N area and leaf turnover rate with high disturbance rate (Disturbance mean interval 5). The size of the point represent the variation of LMA.}
\label{fig:lma_nareaFh2}}
\end{figure}




\clearpage

\section{Discussion}


\begin{figure}[ht]
\centering
\includegraphics{../figures/data_lma_narea.pdf}
\caption{\textbf{Observed within site SMA relation between $LMA$ and
    $N_{area}$ in glopnet data \citep{Wright-2004} for sites with at
    least 10 observations.}
\label{fig:lma_narea_glopnet}}
\end{figure}

\begin{itemize}

\item Negative correlation emerge between $LMA$ and $N_{area}$ but
  this correlation is less clear when the additional cost of $N_{area}$
  on leaf turnover rate is added. In contrast \citet{Wright-2004}
  reported a weak but positive correlation between $LMA$ and
  $N_{area}$ (see Figure in Figure. \ref{fig:lma_narea_glopnet}). The
  negative correlation predicted by \plant\ can be inversed or
 blurred when a $N_{area}$ cost in term of leaf life span is
 included. WHY high $N_{area}$ even more counter selected in shade
 because of additional cost on top of respiration. To compensate for
 the leaf turnover cost of high $N_{are}$ selection for high $LMA$.
  
\item \textbf{Without leaf turnover cost of $N_{area}$.} With
  decreasing vpd there is a shift toward higher values of $LMA$ and
  $N_{area}$ that mainly occurs because of an elimination of low
  values but not an increase in the maximum values. This pattern of
  counter selection of low traits value at high vpd is stronger for
  $N_{area}$ than for $LMA$. This is in agreement with the pattern
  observed in glopnet \citep{Wright-2004} with a counter selection of
  low value of $LMA$ and $N_{area}$ at low vpd. The range of value
  predicted for $N_{area}$ is close to the range observed and the
  range of value predicted for $LMA$ is close but with a tendency to
  predict higher values above 0.1.  

\item \textbf{Without leaf turnover cost of $N_{area}$.} With
  increasing vpd there is a decrease of the range of both $LMA$ and
  $N_{area}$ and smaller number of 'species' coexisting. The range and
  number of species is larger when the disturbance return interval is
  long as already shown by \citet{Falster-2017}. This is in agreement with the pattern observed in glopnet \citep{Wright-2004}.

\item \textbf{With leaf turnover cost of $N_{area}$.} With increasing
  vpd there is a shift toward higher values of $LMA$ because changes
  in both minimal and maximal values. At low vpd high $LMA$ is
  selected to compensate the fast leaf turnover rate due to high
  $N_{area}$. Why selection against high $LMA$ without vpd stress???
  (Because low $N_{area}$ rresults in slow turnover rate even for low
  $LMA$, low construction cost). $N_{area}$ changes are stronger
  than without LTR cost but are still resulting mainly from a change
  of minimal values. 

\item \textbf{With leaf turnover cost of $N_{area}$.} With increasing
  vpd the range of $LMA$ become extremely small. This could be because
  the range of viable $LMA$ is reduced at very high leaf turnover. The small range of
  value for $LMA$ is not in agreement with the pattern observed in
  glopnet \citep{Wright-2004}. 

\end{itemize}


\clearpage

\bibliographystyle{amnat}
\bibliography{references}
\end{document}
