\documentclass[a4paper,11pt]{article}
\usepackage[osf]{mathpazo}
\usepackage{ms}
\usepackage{natbib}
\usepackage{graphicx}
\usepackage{caption}
\usepackage{hyperref}
%% \usepackage{authblk}
\usepackage[labelfont=bf]{caption} % make label for figure bold

% Allow referencing into the supporting information, once that exists.
\IfFileExists{./competition-kernels-sm.tex}{%
  \usepackage{xr}%
  \externaldocument{competition-kernels-sm}}{}

% We will generate all images so they have a width \maxwidth. This means
% that they will get their normal width if they fit onto the page, but
% are scaled down if they would overflow the MAPgins.
\makeatletter
\def\maxwidth{\ifdim\Gin@nat@width>\linewidth\linewidth\else\Gin@nat@width\fi}
\def\maxheight{\ifdim\Gin@nat@height>\textheight\textheight\else\Gin@nat@height\fi}
\makeatother
\setkeys{Gin}{width=\maxwidth,height=\maxheight,keepaspectratio}

\title{Leaf N along water stress gradient}

\author{Georges Kunstler, Daniel S. Falster, Richard G. FitzJohn}
\date{}
\affiliation{Irstea, Grenoble, France and Department of Biological Sciences, Macquarie University,
  Sydney, Australia}
\date{}
\runninghead{}
\keywords{}

\usepackage{color}

\input{common-defs}

\begin{document}

\mstitleshort
%% \mstitlepage
\parindent=1.5em
\addtolength{\parskip}{.3em}

% \begin{abstract}
% Abstract goes here\ldots
% \end{abstract}

\section{Background \& Motivation}

Understanding changes in vegetation composition across large biogeographic gradients is fundamental because vegetation is key to control ecosystem functioning, such as net primary production, and carbon storage. These changes in composition are also fundamental to understand biodiversity, as turnover in species composition explain a large proportion of the global diversity ($\beta$ diversity). In addition, these patterns are also the outcomes of long-term evolutionary adaptation. Understanding the drivers of these patterns is important not only for its intrinsic interest, but also as a path towards understanding how vegetation may change as result of environmental change.

Beyond species succession pattern along climatic gradients, several
studies document gradient of functional traits along climatic
gradients. One specificity of observed pattern of traits variation is
that they are generally weak with generally a higher variation within
sites than along the climatic gradients \citep[see][]{Wright-2004}. A
classical example is the variation of leaf mass per area ($LMA$) and
leaf nitrogen per area ($N_{area}$) with aridity (see literature.pdf
for a in progress review of the literature). $LMA$
\citep{Wright-2004,Onoda-2011,Moles-2014} and $N_{area}$
\citep{Wright-2005,Maire-2015} both increase with aridity, but these
patterns are very weak.


Plot of pattern for leaf traits extracted from glopnet \citep{Wright-2004}.

\begin{figure}[ht]
\centering
\includegraphics{../figures/leaf_climate.pdf}
\caption{\textbf{Pattern in leaf mass per area and leaf nitrogen per area along water availability index (a and d) and MAT over MAP (b and e), and MAP over PET (c and f).} Data are from glopnet \citep{Wright-2004} and MAP over PET is from \url{http://www.cgiar-csi.org/data/global-aridity-and-pet-database}.
\label{fig:leafpattern}}
\end{figure}

\clearpage

\section{Potential mechanisms for these patterns}

Documenting empirical patterns of variation of plant traits along
climatic gradient represents a key first step to understand changes in
vegetation composition, but we also need to propose mechanisms to
explain these traits patterns. Several approach have been proposed to
understand these mechanisms. First, there is along history of studies
proposing adaptative arguments based on an optimization approach with
ecophysiological models of plant functioning \citep{Makela-2002}.

For instance, high $LMA$ has been proposed as an advantage in dry climate because
stiffer leaves can have a higher resistance to dehydration
\citep{Wright-2002a,Wright-2002b} or the need to avoid overheating
under condition of low transpiration \citep{Leigh-2012}. Higher
$N_{area}$ in dry climate have also been proposed as strategy to
maximise photosynthesis \citep{Wright-2003}. This
based on the idea that plant are minimizing the summed cost of water
loss and carbon gain
\citep{Medlyn-2002,Wright-2003,Prentice-2014,Lu-2016,Wang-2017,Dong-2017}. As
higher $N_{area}$ translate into higher $V_{cmax}$ this allow the
plant to achieve a high $A_{area}$ even at the low stomatal conductance
$g_s$ encountred in dry climate.

These approaches represent a key first step, but
they provide no explanation to the large trait variation observed
within each site. This is because they fail to account for the fact that
these traits also controls species interaction and the coexistence of
species with different traits values. A much more abstract approach
have used theoretical models to explore how abiotic constrains and competitive interactions drive species evolution and
changes in traits composition along theoretical abiotic gradients
\citep{Case-2000,Doebeli-2003,Goldberg-2006,Leimar-2008}. These
studies have shown that competition can affect the range of traits
value found in a given community and make the traits deviate from the
theoretical climatic optimum.  Generally, one single traits determines
both the local adaption to the abiotic conditions -- through the
distance to an optimal trait value varying along the gradients -- and
the competitive interaction -- through a normal competition kernel
based on trait value of the species and its competitors
\citep[see][]{Case-2000}. But because these studies are based on abstract traits, they are of limited use to explain observed patterns in traits variation in the field.

Recently a few studies have started to connect physiological approach
and the theoretical approach to build trait-based models of full
communities dynamics either at local \citep{Farrior-2013} or global
scale \citep[see][]{Sakschewski-2015,Scheiter-2013}. They represent key elements of plant ecophysiological functioning. But they have been largely unsuccessful at explaining the wide range of traits found in a single community as these models generally failed to capture the processes promoting species coexistence.

It is thus key to develop mechanistic models where traits are related
to both the direct climate effect and to species coexistence. To do
this for $LMA$ and $N_{area}$ we use a new model -- \plant\
\citep{Falster-2016,Falster-2017} -- that connects directly with leaf
traits and simultaneously allows for coexistence within sites and
variation across sites. Coexistence is driven by competition for light
in size-structured meta-populations and the outcomes of competition is
moderated by environmental factors such as site productivity and
disturbance regime. In \plant\ the coexistence between tree is mainly related to difference in successional niche underpinned by $LMA$ through competition for light. The coordination between $LMA$ and $LSS$ discriminate different strategies of resource use. Species with low $LMA$ have a greater potential maximum growth and are thus favored in high light condition, whereas high $LMA$ species have longer leaf life span and thus are more conservative of their resource and are favored in low light condition.

\section{Key questions}

We will explore how the mixture of $LMA$ and $N_{area}$ predicted by \plant\ changes along aridity gradient exploring three different assumptions about how aridity
affect the dynamics and competition for light.

\begin{itemize}

\item First focusing only on $LMA$, the aridity gradient can simply affect
the productivity with consequence on the process of light
competition thus changing the evolved mixture of $LMA$ (as already shown by \citep{Falster-2017}). This would allow to explore the idea that high $LMA$ is favoured at high aridity because it lead to slow leaf turnover \citep{Wright-2002b}.

\item Secondly, the environmental gradient can affect the
trade-off underpinning the light use strategy (by changing the
trade-off between LMA and LLS). For instance, it has been reported
that the coordination between $LLS$ and $LMA$ is changing along
aridity gradient both in regional \citep{Wright-2002b} and global
studies \citep{Wright-2004}. The elevation of the relation between
$LMA$ and $LLS$ decreases with aridity resulting in a lower leaf life
span for a given $LMA$. This could be due to a higher
investment in leaf nitrogen per area $N_{area}$ at higher aridity
\citep{Wright-2002b} that would results in a decrease leaf life span. Some studies have also reported potential
effect on the slope of the relationship. We will thus explore how the
change in the trade-off between $LMA$ and $LLS$ can influence shift in
trait assembly predicted by \plant\.

\item Finally, we could explore the role of $N_{area}$, which can have
  direct influence on the species tolerance to aridity as proposed by
  \citet{Wright-2003}. Briefly, $N_{area}$ is strongly related to the
  abundance of photosynthetic apparatus in the leaf and allow the
  plant to achieve higher photosynthesis for a given stomatal
  conductance. Higher $N_{area}$ come however with the cost of a high
  respiration rate. We will use \plant\ to predict assembly of these
  two traits along an aridity gradient with two approaches. First
  using an empirical photosynthesis model with a theoretical effect of
  water stress and $N_{area}$ on productivity. Secondly, using the
  widely used Farqhuar model connected to $N_{area}$ and a stomatal
  conductance model responding along a vpd gradient. The first
  approach is general but the relative effect of $N_{area}$ and
  aridity are not predicted from first principle. The second approach
  is based on well know mechanisms implemented in most DGVMs but
  represent only a part of aridity effect through vpd.

% and with a coupled model of photosynthesis and stomatal conductance (see \citet{Prentice-2014,Wang-2017} for optimization approach using this type of models).

\end{itemize}



% \subsection{Review of optimal approach for $LMA$ and $N_{area}$ response to aridity}

% High $LMA$ has been proposed as an advantage in dry climate because
% stiffer leaves can have a higher resistance to dehydration
% \citep{Wright-2002a,Wright-2002b} or the need to avoid overheating
% under condition of low transpiration \citep{Leigh-2012}. (Other refs ?)

% Optimization literature of $N_{area}$ predict that $N_{area}$ should
% increase with aridity \citep{Wright-2003}. The general approach is
% based on the idea that plant are minimizing the summed cost of water
% loss and carbon gain
% \citep{Medlyn-2002,Wright-2003,Prentice-2014,Lu-2016,Wang-2017,Dong-2017}. As
% higher $N_{area}$ translate into higher $V_{cmax}$ this allow the
% plant to achieve a high $A_{area}$ even at low stomatal conductance
% $g_s$. Recently \citet{Prentice-2014} and \citet{Dong-2017} provided
% quantitative predictions for the effect of key climatic variable
% (including VPD) based on an optimized Farquhar photosynthesis
% model. Several authors have challenged the idea of summing the cost of
% water and carbon gain on the ground that competition should favor
% carbon gain optimization only \citep{Wolf-2016} and the idea that the
% main effect of carbon stress come through cavitation \citep{Sperry-2016,Sperry-2017}.

% A key point that have been raised by \citet{Dong-2017} is that
% $N_{area}$ could be divided in structural and rubisco related N (using
% the relationship of \citet{Onoda-2004}). {\color{red}TO DISCUSS WITH DANIEL SPLIT in $N_{STRUCT}$ VS $N_{RUBISCO}$ AND HAVE N STRUCT LINK TO LMA??}


\section{Overview of analysis and first results}


\subsection{$LMA$ only and productivity gradient}

In a first set of assumption we consider a model where there is only one trait LMA. In the most simple version environmental gradients changes productivity (photosynthesis $A_{max}$ through a linear effect of a theoretical productivity gradient $p$) and
thus competition for light and selection on LMA (this is already presented in \citet{Falster-2017}) (high LMA as adaptation to both log light and low productivity).

The figure \ref{fig:lma} represents how the evolved community mixture
of LMA change along a productivity gradient. This is essentially
representing the results of \citet{Falster-2017} only over LMA
(leaving out size at maturity). Low LMA strategies, coressponding to
fast-growing light demanding species are only possible in productive sites.

\begin{figure}[ht]
\centering
\includegraphics{../figures/gradient_lma_multi.pdf}
\caption{\textbf{Predicted community assembly of leaf mass per area along a productivity gradient controlled by aridity by 'Plant' \citep[see][]{Falster-2016}}
\label{fig:lma}}
\end{figure}

\clearpage

\subsection{$LMA$ only and changes in $LMA$ \textit{vs.} $LTR$}

In second version, the aridity gradient affect the productivity ($A_{max}$) but also change the trade-off underpinning LMA-LLS in term of its elevation or its slope.

\citet{Wright-2005} reported that the trade-off between LMA and Leaf Turnover Rate, LTR ($1/LLS$), is changing with aridity with higher aridity sites having higher LTR at a given LMA and a shallower slope. \footnote{There is also a shift in the link $A_{mass}$ and LMA (with higher in dry sites), probably because leaf N, but in a first step we will ignore that as the current version consider $A_{area}$ constant and the photosynthesis integrated over the year (as implemented in \plant\ ) is probably lower in dry sites. So we will keep the link between $a_{p1}$ and stress gradient as above. Note that \citet{Sakschewski-2015} implement a link between LMA and $v_{cmax_{area}}$.}

This can be seen In Glopnet as (i) the elevation of the LTR \textit{vs.} LMA is changing with mean annual precipitation (MAP) (see Figure \ref{fig:MAP}) and (ii) the slope of the LTR \textit{vs.} LMA is changing with the ratio MAT/MAP (see Figure \ref{fig:MAT_MAP}). To implement these two distinct options in `plant` we first regressed the elevation and the slope of the site level LTR \textit{vs.} LMA relationship against MAP or MAT/MAP and then included this in `plant` (see Figure \ref{fig:elev_slope}). One key issue here is how to scale the theoretical aridity gradient we used in the previous model ($A{max}_{area}$ is a linear function with slope one of this gradient) with MAP or MAT/MAP. \footnote{So far I have just assumed that the range of MAP or MAT/MAP analyzed was restricted to the part of the theoretical gradient where growth was positive (above -0.3) and thus re-scaled the range of MAP or MAT/MAP to a variation of $p$ from -0.3 to 0.3. This is clearly not perfect.}

\begin{figure}[ht]
\centering
\includegraphics{../figures/data_lma_ll_trade_off_climate_lev_P.pdf}
\caption{\textbf{LTR vs LMA tradeoff per classes of MAP}
\label{fig:MAP}}
\end{figure}


\begin{figure}[ht]
\centering
\includegraphics{../figures/data_lma_ll_trade_off_climate_lev_T_P.pdf}
\caption{\textbf{LTR vs LMA tradeoff per classes of MAT/MAP}
\label{fig:MAT_MAP}}
\end{figure}


\begin{figure}[ht]
\centering
\includegraphics{../figures/data_lma_ll_trade_off_climate_slope_elev.pdf}
\caption{\textbf{Regression of site level elevation and slope parameters of the LTR \textit{vs.} LMA relationship (estimated with smatr::sma) against MAP or MAT/MAP. }
\label{fig:elev_slope}}
\end{figure}


%% This variation may be related to a change in Leaf N, with species from lower precipitation sites having higher leaf $N_{mass}$ resulting in lower tissue toughness but higher $A_{mass}$ \citep{Wright-2002a}. In term in $A_{area}$ this would translate in higher $A_{area}$ at low precipitation sites.  This is a change in optimum $A_{area}$ the $A_{area}$ integrated over the year (as implemented in \plant\ ) is probably higher in site with high precipitation.


The figure \ref{fig:lma_map} represents how the evolved community
mixture of LMA change along a productivity gradient when the elevation
and slope (but mainly elevation) of LMA vs LTR were changing with
precipitation MAP.

\begin{figure}[ht]
\centering
\includegraphics{../figures/gradient_lma_multi_elev.pdf}
\caption{\textbf{Predicted community assembly of leaf mass per area
    along a productivity gradient controlled by aridity by 'Plant'
    with a variation of LTR \textit{vs.} LMA relationship with precipitation (mainly elevation).}
\label{fig:lma_map}}
\end{figure}

The figure \ref{fig:lma_mat_o_map} represents how the evolved
community mixture of LMA change along a productivity gradient when the
elevation and slope of LMA vs LTR were changing with MAT/MAP. In both case the gradient of $LMA$ is steeper than
with a constant $LMA$ \textit{vs.} $LTR$ relationship. With MAP (Figure \ref{fig:lma_map}) the $LMA$ is
higher at low productivity and lower at high productivity than with a
constant $LMA$ \textit{vs.} $LTR$ relationship. With MAT over MAP this
is mainly lower value at high productivity.

\begin{figure}[ht]
\centering
\includegraphics{../figures/gradient_lma_multi_slope.pdf}
\caption{\textbf{Predicted community assembly of leaf mass per area
    along a productivity gradient controlled by aridity by 'Plant'
    with a variation of LTR \textit{vs.} LMA relationship with MAT/MAP
    (mainly slope).}
\label{fig:lma_mat_o_map}}
\end{figure}


\clearpage

\subsection{$LMA$ and $N_{area}$}

A third set of models could explore how LMA interact with Leaf N, a second trait controlling the response to the environmental gradient. Aridity gradients affect productivity (photosynthesis) but a
high leaf N allow to maintain a high photosynthesis even at low
stomatal conductance (so in condition of high aridity). This is
however coming with a higher respiration cost (potentially also a higher leaf turnover \citep{Wright-2002a,Wright-2002b}, but this is not modelled in plant). This may results in LMA shifting less than would have been otherwise, but instead differentiate species in N (with flow on effect to shade tolerance).


 \citet{Wright-2003} explore how N can substitute for water. Assuming that $A_{area} \propto N_{area} \, g_s$ (with $g_s$ stomatal conductance, and $N_{area}$ being related to $V_{cmax}$ and thus $c_a - c_i = c(1-\chi)$). One solution would be to assume that the water stress is represented by $1/g_s$ and $N_{area}$ could off-set the effect of water stress on $A_{area}$ (in our case $A_{max}$). The cost of leaf N comes from the higher dark leaf respiration per area for higher $N_{area}$ and is already include in \plant\ . \footnote{Prentice et al. 2014 expanded this prediction in term of an optimization with a details Farquhar model for C3 plant, but focusing directly on $V_{cmax}$ rather than $N_{area}$.}

 The current implementation of Leaf nitrogen per area in `plant` already captures some of the effect proposed by \citet{Wright-2003}. This is build on an empirical relationship between $A_{max}$ and $N_{area}$
($A_{max} = b_{lf1} \times \frac{N_{area}}{N^0_{area}}^{b_{lf5}}$ \citep{Falster-2016}). There is however no quantitative prediction of how water stress influence $A_{max}$ for different level of $N_{area}$. A first option would be to simply consider that $A_{max}$ is linearly influenced by a water stress index  as follow $A_{max} = (1+WS) * b_{lf1} \frac{N_{area}}{N^0_{area}}^{b_{lf5}}$ ($WS$ is an aridity index ranging between -0.6 high aridity to 0.6 low aridity). This model is presented in the figure \ref{fig:leafN_water}. This could allow a first exploration of the implication of this assumption on community assembly.


\begin{figure}[ht]
\centering
\includegraphics{../figures/WaterS_LeafN_contour.pdf}
\caption{\textbf{Variation of $A{max}_{area}$ with leaf nitrogen per area and water availability (productivity gradient) predicted by `plant`.}
\label{fig:leafN_water}}
\end{figure}

The figures \ref{fig:lma_narea1} and \ref{fig:lma_narea2} show that
the gradient of $LMA$ is weaker than in the previous models but that a gradient of $N_{area}$ is predicted. There is also a negative correlation between $N_{area}$ and $LMA$ within sites.

\begin{figure}[ht]
\centering
\includegraphics{../figures/gradient_narea_lma_multi_narea_lma2.pdf}
\caption{\textbf{Predicted community assembly of leaf mass per area and leaf N per area along a productivity gradient controlled by aridity by 'Plant'.}
\label{fig:lma_narea1}}
\end{figure}

\begin{figure}[ht]
\centering
\includegraphics{../figures/gradient_narea_lma_multi_narea_lma.pdf}
\caption{\textbf{Predicted community assembly of leaf mass per area and leaf N per area along a productivity gradient controlled by aridity by 'Plant'. The size of the point represent the variation of LMA.}
\label{fig:lma_narea2}}
\end{figure}

\clearpage

\section{Farquhar photosynthesis model, water stress, and Leaf N}


It would be interesting to explore if using a mechanistic model for
photosynthesis and water stress could lead to quantitative predictions
about how water stress influence growth and how $N_{area}$ influences the response to water stress. I have started to explore this below.


\subsection{Farquhar photosynthesis model and plant photosynthesis model.}

The FvC model describe leaf photosynthesis is describe as the minimum
of three potential rates $J_E$ the light limited rate, $J_C$ the CO2
rubisco limited rate, and $J_S$ the starch export limitation rate. The
last limitation by starch export (or triose phosphate utilization)
limitation is generally neglected as it come into play only at very
high level of $CO_2$.

\begin{equation}
\label{eq:An}
A_n= min(J_E, J_C, J_s) - R_d.
\end{equation}

%% There is a lot of variant along this model but
%% overall they would all predict that $N_{area}$ would influence
%% $A_{max}$ by influencing either $J_C$ which is a function of $V_{cmax}$
%% which is related to $N_{area}$ or $J_E$ which is related to $J_{max}$
%% which related to $N_{area}$ as well. It is probably better to use
%% hyperbolic minimum rather than minimum function to avoid
%% discontinuity.

%% The key mechanisms which support that $N_{area}$ might allow to
%% adapt to low water availability is that increasing $N_{area}$ at a
%% given stomatal conductance $g_s$
%% increase $V_{cmax}$ and thus decrease the ratio $c_i/c_a$.


%% \begin{equation}
%% \label{eq:JC}
%% J_E = Q_p \frac{a \alpha (p_i - \Gamma_*)}{p_i + 2 \Gamma_*}.
%% \end{equation}

%% , and $Q_p$ is the incident flux of photo-synthetically active photon.

\begin{equation}
\label{eq:JC}
J_C= \frac{V_{cmax} (p_i - \Gamma_*)}{p_i + K_c (1+[o_2]/K_o)}.
\end{equation}

where $K_c$ and $K_o$ are the Michaelis constant for $CO_2$ and the competitive inhibition constant for $O_2$, and $V_{cmax}$ is the maximum catalytic capacity of Rubisco per leaf area.
$p_i$ is given by $p_i = P c_i$. And $\Gamma_* = \frac{[O_2]}{2\tau}$ ($\tau$ is a ratio of kinetic parameters describing the partitioning of RuNP to the carboxylase or oxygenase)

\begin{equation}
\label{eq:JCb}
J_E = J \frac{ c_i - \Gamma_*}{4c_i + 8 \Gamma_*}.
\end{equation}

where

\begin{equation}
\label{eq:Jlight}
J = \frac{ \alpha I + J_{max} - \sqrt{(\alpha I + J_{max})^2 - 4 \theta \alpha I J_{max}}}{2\theta}.
\end{equation}

with $\alpha$ the initial quantum yield (in mol of electron per mol of photon), $\theta$ the curvature of the light response and $J_{max}$ the maximum electron transfer,
see for instance \citet{Bernacchi-2009}.


\begin{equation}
\label{eq:JS}
J_S= V_{cmax}/2.
\end{equation}

To have a more gradual transition than the minimum function it is possible to use the smaller roots of two quadratics functions:

\begin{equation}
\label{eq:Q1}
\theta J_p^2 - J_p(J_E+J_C) + J_E J_C= 0.
\end{equation}

where $J_P$ is the min of $J_C$ and $J_E$.

\begin{equation}
\label{eq:Q2}
\beta A^2 - A(J_P+J_S) + J_P J_S= 0.
\end{equation}

where $\theta$ and $\beta$ are constant describing the transition between limitation.

The kinetic parameters $K_c$, $K_o$, $V_{cmax}$, and $\tau$ change with temperature according to a $Q_{10}$ function ($k = k_{25} Q_{10}^{(T-25)/10}$).

\clearpage

We can use the R package plantecophys \citep{Duursma-2015} to compare
the current photosynthesis model of \plant\ and the FvC model for
different parameters. In most of the cases I assumed as in
\citet{Medlyn-2002} that $J_{max} = 1.67 \times V_{cmax}$. Several
other equations have been proposed see
\citet{Walker-2014,Kattge-2011}.
% For the simulations with $N_{area}$ we will also need to express $V_{cmax}$ in function of $N_{area}$ \citep{Kattge-2009,Domingues-2010,Sakschewski-2015}. For instance, \citet{Sakschewski-2015} fitted the following relationship on the TRY data $V_{cmax25} = 31.62 N_{area}^0.801$.


\begin{figure}[ht]
\centering
\includegraphics{../figures/Photo_Plant_FvC.pdf}
\caption{\textbf{Photosynthesis model of \plant\ and FvC for different parameterisation.}
\label{fig:photo}}
\end{figure}

The figure \ref{fig:photo} show the light response curves for \plant\
, the default FvC, FvC with the light parameters similar to \plant\ , the parameters used in \citet{Sterck-2011}, and the parameters used in Troll \citep{Marechaux-2017}.

Then we can see how this scale up to the annual photosynthesis rate using the integration approach of \plant\ . \plant\ use a Michaelis Menten function to approximate the annual photosynthesis rate see figure \ref{fig:photo_annu_plant}. The Michaelis Menten approximation is not working well for the FvC model, see Figure \ref{fig:photo_annu_fvc}. An exponential function \citep{Chen-2016} ($A = p_1 *(1-e^{-p_2* E/p_1})$) or the non-rectangular hyperbola work better, see figure \ref{fig:photo_annu_fvc}.


\begin{figure}[ht]
\centering
\includegraphics{../figures/Annual_Photo_Plant.pdf}
\caption{\textbf{Annual photosynthesis model of \plant\ .}
\label{fig:photo_annu_plant}}
\end{figure}


\begin{figure}[ht]
\centering
\includegraphics{../figures/Annual_Photo_FvC_vpd0.pdf}
\caption{\textbf{Annual photosynthesis model of FvC with the parametrisation of Troll, $\alpha = 0.3$, $\theta = 0.7$, $V_{cmax} = 38$ and $J_{max} = 1.67 * V_{cmax}$.}
\label{fig:photo_annu_fvc}}
\end{figure}

\clearpage

\subsection{Coupled photosynthesis and water stress models}

To include the effect of aridity it is need to couple the photosynthesis model with a stomatal conductance model. In most approaches the stomatal conductance model is linked with FvC model with the Fick law:

\begin{equation}
\label{eq:fick}
A_n = g_s (c_a - c_i) = g_s c_a (1-\chi).
\end{equation}

Where $c_a$ and $c_i$ are respectively the $CO_2$ mole fraction of the air and in the leaf.


For the stomatal model a lot of different options have been proposed that range from empirical models to optimization models and from models that include or not soil water stress. If soil water stress is include some approach also include competition for soil water between trees \citep{Farrior-2013}. My general idea would be to start by a relatively simple model for a first paper and then build on to include more complex processes. In the first model there would be not effect of soil water stress and no representation of competition for water.

I will give a brief overview of the existing stomatal models.

One of the first and most classical model is the one of \citet{Collatz-1991} which connect the FvC photosynthesis model with the Ball Berry model which predict $g_s$. This is an empirical relationship based on the following equation:

\begin{equation}
\label{eq:gs-Ball}
g_s= g_0 + g_1 \frac{A_n h_s}{c_a}.
\end{equation}

where $h_s$ relative humidity at the leaf surface, and $c_a$ $CO_2$ mole fraction of the air (at leaf surface).

This has been modified by \citet{Leuning-1995} as the following empirical equation:

\begin{equation}
\label{eq:gs-Leuning}
g_s= g_0 + g_1 \frac{A_n }{(c_a - \Gamma)(1+D/D_0)}.
\end{equation}

Where $D$ is the air vapor pressure deficit (VPD).

Then several approach have been developed to predict stomatal conductance based on the optimization of the summed cost of water loss and carbon gain \citep{Medlyn-2002,Prentice-2014,Wolf-2016,Sperry-2017}. One of the most used approach is the one developed by \citet{Medlyn-2011} as :

\begin{equation}
\label{eq:gs-Medlyn}
g_s \approx g_0 + (1 + \frac{g_1}{\sqrt{D}}) \frac{A_n }{c_a}.
\end{equation}


Then other authors have proposed to account for the impact of xylem cavitation \citet{Wolf-2016,Sperry-2016,Sperry-2017} in the model and also to maximize carbon gain rather than the water use efficiency \citet{Wolf-2016}. I think that \citep{Sperry-2016} propose a framework that would be easy to implement and to connect to a water budget in \plant\ .

But as an initial step we could use either the model of \citet{Medlyn-2011} or \citet{Leuning-1995}. We need to keep in mind that these models that are based only on vpd are supposed to be applicable only to plant without soil water stress.

The figures \ref{fig:photo_stomat}, \ref{fig:photo_stomat_BB}, \ref{fig:photo_stomat_Leuning}, and \ref{fig:photo_stomat_opti} show the different coupled models in plantecophys.

\begin{figure}[ht]
\centering
\includegraphics{../figures/FvC_stomatal_BallBerry.pdf}
\caption{\textbf{Coupled FvC and Ball Berry models with effect of $V_{cmax}$ and VPD.}
\label{fig:photo_stomat_BB}}
\end{figure}

\begin{figure}[ht]
\centering
\includegraphics{../figures/FvC_stomatal_BBLeuning.pdf}
\caption{\textbf{Coupled FvC and Leuning models with effect of $V_{cmax}$ and VPD.}
\label{fig:photo_stomat_Leuning}}
\end{figure}

\begin{figure}[ht]
\centering
\includegraphics{../figures/FvC_stomatal_BBOpti.pdf}
\caption{\textbf{Coupled FvC and Medlyn models with effect of $V_{cmax}$ and VPD.}
\label{fig:photo_stomat_opti}}
\end{figure}

\begin{figure}[ht]
\centering
\includegraphics{../figures/FvCstomatal.pdf}
\caption{\textbf{Coupled FvC and Medlyn models with effect of $V_{cmax}$ and VPD.}
\label{fig:photo_stomat}}
\end{figure}

\clearpage

The variation of the Leuning model with vpd seems to give rise to an annual model that is well approximated by the non-rectangular hyperbola (see \ref{fig:photo_annu_fvc_vpd3}).

\begin{figure}[ht]
\centering
\includegraphics{../figures/Annual_Photo_FvC_vpd3.pdf}
\caption{\textbf{Annual photosynthesis model of FvC with the parametrisation of Troll, $\alpha = 0.3$, $\theta = 0.7$, $V_{cmax} = 38$ and $J_{max} = 1.67 * V_{cmax}$ coupled with the Leuning stomatal model and integrated over one year in the left panel.}
\label{fig:photo_annu_fvc_vpd3}}
\end{figure}

So my proposition would be to run simulation with this coupled FvC-Leuning model along a gradient of vpd where the $V_{cmax}$ and $J_{max}$ are related to $N_{area}$ using the link proposed by \citep{Sakschewski-2015}.

\subsection{Parametrisation of the FvCB}


Choice of parameters of the FvCB model to use for
forest trees? Here is a short list of existing parametrisation for the
light response part:
\begin{itemize}

\item In plantecophys \citep{Duursma-2015}  $\alpha = 0.24$ and $\theta = 0.85$

\item In \citet{Medlyn-2002} $\alpha = 0.3$ and $\theta = 0.9$

\item In Plant $\theta = 0.5$ and $\alpha= 0.04$ not on the same unite
  (mol of CO2 per mol of photon wherease in FvCB in mol electron per
  mol photon)  (conversion see Mercado et al 2009, the conversion of
  the apparent quantum yield in micromolCO2/micromol quantum into
  $\mu$ mol e-/micxromol quantum is done by multipliyng by 4, since
  four electrons are needed to regenerate RuBP $4*0.04 = 0.16$ but we
  also probably need to correct for leaf absorbance for instance 0.8 in \citet{Medlyn-2002})

\item In \citet{Sterck-2011} $\theta = 0.5$ and $\alpha = 0.25$ (not exactly the same alpha ...)

\item  In \citet{Valladares-1997} for tropical shrubs $\theta$ ranges between 0.51 and 0.8 and $\alpha$ (in mol CO2) between 0.048 and 0.066

\item In TROLL \citet{Marechaux-2017} $\alpha = 0.075*4 = 0.3$ and $\theta = 0.7$ based on von Caemmerer 2000

\item In \citet{vonCaemmerer-2000} the non rectangular hyperbola light response curve has an initial slope $\alpha_1 \times \Phi_{PSII} \times \beta$ with $\alpha_1$ the leaf absorptance, $\Phi_{PSII}$ the maximum quantum yield of the photosystem II and $\beta$ the fraction of absorbed light that reach PSII.

\end{itemize}

I propose to use the parametrisation of \citet{Marechaux-2017}.


\subsection{Parametrisation of Jmax \textit{vs.} Vcmax}

The hypothesis of a coordination between Calvin-Benson cycle limited
rate of assimilation and electron transport limited rate of
assimilation, lead to the asumption of a correlation between Vcmax and
Jmax. Different model to link them have been proposed in the
litterature.

\begin{itemize}

\item In \citet{Medlyn-2002} $J_{max} = 1.67 \times  V_{cmax}$.

\item In \citet{Walker-2014} the relationship is a power function of $V_{cmax}$ $J_{max} = exp(1.01) \times V_{cmax}^{0.89}$.

\item In Try data from \citet{Kattge-2011} this is $J_{max} = exp(1.669) \times V_{cmax}^{0.75}$.

\item In \citet{Wullschleger-1993} this is $J_{max} = exp(1.425) \times V_{cmax}^{0.837}$.

\end{itemize}

The Figure \ref{fig:Vcmax_Jmax} show the different parametrization.

\begin{figure}[ht]
\centering
\includegraphics{../figures/Param_FvC_Vcmax_Jmax.pdf}
\caption{\textbf{Different parametrization of the link between $J_{max}$ and $V_{cmax}$.}
\label{fig:Vcmax_Jmax}}
\end{figure}

I propose to use either \citet{Medlyn-2002} or \citet{Walker-2014}.

The slope of Jmax vs Vcamx (in log log) affect the light level at which photosynthesis become RuBisCO limited. A shallower slope shift the light level to higher light value and a steeper slope to a lower light value. This will thus affect shade tolerance.

The link between Vcmax and Jmax might however be more complexe and depend on the light level and leaf P. According to Walker et al. 2014 the effect of SLA and leaf P are however relatively small. So this is ok to use only Jmax in function of Vcmax.

\subsection{Parametrisation of Vcmax \textit{vs.} Narea}

Photosynthetic rate should scale positively with leaf Nitrogen (N) because of teh large amount of N invested in RuBisCO. Several relationship have been proposed to link $V_{cmax}$ and $N_{area}$.

\begin{itemize}

\item \citet{Sakschewski-2015} proposed that $V_{cmax} = 31.62 \times N_{area}^{0.801}$ (unit g per m2 need to be converted in kg per m2 as $*1000^{0.801}$.

\item  \citet{Walker-2014} proposed that $ln(Vcmax) = 3.946 + 0.921 \times log(N) + 0.121 \times log(P) + 0.281 \times log(N) \times log(P)$ and explored other model that include also LMA. To get a function of only N this should be $V_{cmax} = exp(3.712) \times N^{0.650}$ see  \citet{Walker-2017}.

\item \citet{Kattge-2011} proposed $V_{cmax} = a + b \times N_{area}$ with different value per biomes estimated either with $V_{cmax}$ only or also constrained by $A_{max}$.

\item \citet{Domingues-2010} proposed several model based on N and P.  If we look only at the one based only on N this is: $V_{cmax} = 10^{1.57} \times N_{area}^{0.55}$ (Appendix C Table A2).

\end{itemize}

The Figure \ref{fig:Narea_Vcmax} show the different parametrization.

\begin{figure}[ht]
\centering
\includegraphics{../figures/Param_FvC_Narea_Vcmax.pdf}
\caption{\textbf{Different parametrization of the link between Narea and Vcmax.}
\label{fig:Narea_Vcmax}}
\end{figure}

Other traits such as leaf P and SLA could also influence Vcmax \citep{Walker-2014,Domingues-2010}. LMA has little effect on the link between Vcmax and Narea.  Leaf P has little direct effect on Vcmax but an important interaction with Narea: at low leaf P the slope of Vcmax \textit{vs} Narea is shallow but at high leaf P this slope is steeper \citep{Walker-2014}.
It is also possible to use an other approach where both Vcmax and Jmax
are function of N and P rather than having Jmax a function of Vcmax (Domingues et al. 2010, Walker et al. 2014).

I propose to use \citet{Sakschewski-2015}.


\subsection{TODO to run simulation with this FvC model in plant}


In the long-term it would be interesting to think about:

\begin{itemize}

\item predict vpd and a full water budget for the observed data (glopnet) to facilitate comparison with theoretical prediction (building on \citet{Prentice-2014}).

\item using stomatal model that is linked to the soil water potential (as in \citet{Sperry-2016} or
\citet{Sterck-2011})

\item include a simple water budget to model tree competition for water

\end{itemize}

%% TODO a list of different assumption an show some example with plantecophys

%% Then proposition of first approach based on vpd or directly gs?



%% The Fick's law close the system of equation:
%% \begin{equation}
%% \label{eq:fick}
%% g_s=\frac{A_d}{c_a - c_i}.
%% \end{equation}

%% $p_i = P c_i$

%% The models are combined by solving simultaneously for $c_i$ and $g_s$.

%% \citet{Scheiter-2009} use this photsynthesis model (with a min function).

%% Other version of the model consider that instead a of $Q_p$ the light incidence, $J_E$ is a function $J$ the electron transport rate which is it self a function of incident light (with non rectangular hyperbola) \citep{Sharkey-2007}.



%% \subparagraph{\citet{Haxeltine-1996}, photosynthesis model assuming optimal allocation of Leaf N and then all LPJ models}

%% In LPJ no $J_S$ and $V_{cmax}$ is optimised to maximise photosynthesis (thus optimum allocation of Leaf N). This optimum allocation results in equation 14 of \citet{Sitch-2008}.

%% Water stress is tracked by a water balaence model with a demand evapotranspiration $E_d$ and a supply determined $E_s$ by the water availability. Water stress occurs when $E_d > E_s$.

%% Then if non water stress conductance is given by the photosynthesis equation by:

%% \begin{equation}
%% \label{eq:fickLPJ}
%% g_s=g_{min} + \frac{1.6A_d}{c_a - c_i}.
%% \end{equation}

%% and for C3 plant the ratio $c_i / c_a$ is set to 0.7.

%% The eqution \ref{eq:fickLPJ} provides non water stressed $g_s$ which then gives $E_{d}$ by

%% \begin{equation}
%% \label{eq:12Sitch}
%% E=E_p \alpha_m [1 - exp(\frac{- g_s}{g_m})].
%% \end{equation}

%% If the condition are water stressed ($E_{supply} < E_{demand}$ value
%% of $g_s$ is given by the equation \ref{eq:12Sitch} (assuming $E$ is
%% equal to $_{supply}$. Then $g_s$ can be used to solve Fick law and the
%% photosynthesis model.

%% \subparagraph{\citet{Sakschewski-2015}, add Leaf N in LPG photosynthesis deacreasing below optimal Leaf N}

%% $V_{cmax}$ is a power function of Leaf N (a function of SLA , even if
%% not the case in Wright et al. 2004). The photosynthesis model of LPJ
%% is based on optimum allocation to Leaf N (so optimal $V_{cmax opt}$) so
%% the actual $V_{cmax}$ is the munimum between the optimum and the one
%% predicted by leaf N. This is some how strange as the effect of leaf N
%% on $V_{cmax}$ is limited by the optimal value.

%% \subparagraph{\citet{Prentice-2014}, optimal $c_i/c_a$ for $V_{cmax}$ vs $g_s$}

%% Predict optimal $c_i/c_a$ to optimise $V_{cmax}$ and $G_s$ (evaporation) for different $D$ vapour pressure deficit.

%% \subparagraph{\citet{Johnson-1984}, model and its use in more recent model}

%% In this alternative model of photosynthesis, only the response to
%% light is considered (and this match the euqation of the light effect
%% on $J$). This is the model used in 'Plant'.

%% \begin{equation}
%% \label{eq:Johnson}
%% A_{N,PAR} = \frac{ \alpha I + A_{max} - \sqrt{(\alpha I + A_{max})^2 - 4 \theta \alpha I A_{max}}}{2\theta}.
%% \end{equation}

%% \subparagraph{\citet{Farrior-2013}}

%% Use a simpler version of the light response curve (instead of using
%% \citet{Johnson-1984}) they use a simpled staturantion function
%% ($A_L = min(AL, A_{max})$). Then water demande is linearly related to
%% $A_L$ and water supply is related to soil water potential with a two
%% step function no supply if below a treshold, above this treshold
%% increase with soil water potential (see Fig. A2). Soil water content
%% is lenearly related to soil water potential.

%% TODO \citet{Zavala-2005} based on \citet{Leuning-1995} for stomatial
%% conductance (a new version of \citet{Collatz-1991}).

%% \subparagraph{Summary and what to do next}

%% Water availability affect photosynthesis in two way. During the period
%% of non water stress how photosynthesis work (the paper of
%% \citet{Wright-2003} deal with that. \citet{Medlyn-2011} also deal
%% with optimal conductance in soil water limited plant.  For
%% \citet{Prentice-2014} as well, ?, but they also argue that vpd
%% represent the long term water availability.

%% In period of non water stress all plant have non limited supply of
%% water from the soil but the vpd is different inducing different
%% transpiration at the same $g_s$. In that case leaf N area control the
%% water use efficiency. Then water availability control also
%% the duration of the year over which soil water is limiting. For a non water
%% stressed plant, soil water supply limit the photosynthesis by
%% constraining the stomatal conducatance to match the supply.



%% There is two different approaches. In the first one, a water budget model
%% predict when soil water is limiting the value of $g_s$ that match
%% evapotranspiration allowed by the supply. (PhD of Lohier differ in the
%% sens that transpiration is not computed as a function of PET).

%% In the second one there is a direct effect of $A$, $c_a$ and the
%% relative humidity on the $g_s$ but not effect of soil water stress.

%% What we need is to have a Photosynthesis model with light effect and
%% $A_{max}$ related to Leaf N as it is currently the case that would be
%% $A_L$ and then add an effectof water stress that would limit to
%% $A_W$. The $A = min(A_L , A_N)$ thus this is different than affecting
%% directly $A_{max}$ as at low light level $A$ will not be affected.

%% Then we need to have a link between Leaf N and the way plant tolerate
%% water stress


%% One approach is to say that there two A one for non water stressed
%% conditions $A_L$ and one for water stressed conditions $A_W$. The
%% water availability variable is $E_S$ evapotransipiration supply
%% (daily, monthly ???) per unit area of leaf. Assuming non water
%% stressed condition we can
%% compute $A_L = f(V_{cmax}, c_i, I)$ assuming $c_i/c_a = 0.7$ (not
%% perfect as Leaf N shouild affect that and thus when water stress
%% occurs). Based on $A_L$ compute $g_{s pot}$ and $E_S$ per unit area of
%% leaf. Then compute $\omega = min(1, E_S/E_D)$. If $\omega = 1$ then $A
%% = A_L$. If $\omega < 1$ then compute $g_s$ from $E_S$ and then for
%% $c_i$ giving the lowest $A$.



%% I don't think that is necessarely need to have $J_C$ $J_E$ (and a
%% proper light response). Weed need water stress reducing $g_s$ and thus $A$ in function of water stress AET/PET. If we derive this model this give use for each AET a prediction $A$ at different leaf N predicting $V_{cmax}$ as in \citet{Sakschewski-2015}.

%% How to simplify that ??? we can


%% Check correspondance to Plant photsynthesis model which is based on Johnson, I. R. and Thornley, J. H. M. 1984. A model of instantaneous and daily canopy photosynthesis. - Journal of Theoretical Biology 107: 531–545.
%%  and is a non-rectangular hyperbola response to light where we could assume that $A_{max}$ is related to $V_{cmax}$ and thus leaf N.

\clearpage


\bibliographystyle{amnat}
\bibliography{references}
\end{document}
