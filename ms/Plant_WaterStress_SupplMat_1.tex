\documentclass[a4paper,11pt]{article}
\usepackage[osf]{mathpazo}
\usepackage{ms}
\usepackage{natbib}
\usepackage{graphicx}
\usepackage{caption}
\usepackage{hyperref}
%% \usepackage{authblk}
\usepackage[labelfont=bf]{caption} % make label for figure bold

% We will generate all images so they have a width \maxwidth. This means
% that they will get their normal width if they fit onto the page, but
% are scaled down if they would overflow the MAPgins.
\makeatletter
\def\maxwidth{\ifdim\Gin@nat@width>\linewidth\linewidth\else\Gin@nat@width\fi}
\def\maxheight{\ifdim\Gin@nat@height>\textheight\textheight\else\Gin@nat@height\fi}
\makeatother
\setkeys{Gin}{width=\maxwidth,height=\maxheight,keepaspectratio}

\title{Supplementary Materials 1. Additional Results}

\author{Georges Kunstler, Daniel S. Falster, Richard G. FitzJohn}
\date{}
\affiliation{INRAE LESSEM, Grenoble, France and Department of Biological Sciences, Macquarie University,
  Sydney, Australia}
\date{}
\runninghead{}
\keywords{}

\usepackage{color}

\input{common-defs}

\begin{document}

\mstitleshort
%% \mstitlepage
\parindent=1.5em
\addtolength{\parskip}{.3em}

% \begin{abstract}
% Abstract goes here\ldots
% \end{abstract}

\section{Introduction}

This supplementary materials present the results of the community
assembly simulations with a disturbance return interval of 40 years
with the plant model \citep{Falster-2016,Falster-2017}.

\clearpage

\section{Results with FVCB model}
\section{Supplementary Materials Additional Results}


\begin{figure}[ht]
\centering
\includegraphics{../figures/gradient_narea_lma_multi_FvCB_cor.pdf}
\caption{\textbf{LMA vs Leaf N area for FVCB model with low disturbance rate (Disturbance mean interval 40 years). The color of the points represents the vpd.}
\label{fig:lma_narea_cor_multiSM}}
\end{figure}



\begin{figure}[ht]
\centering
\includegraphics{../figures/gradient_narea_lma_multi_narea_lma_FvCB.pdf}
\caption{\textbf{Predicted community assembly of leaf mass per area and leaf N per area along a productivity gradient controlled by aridity by 'Plant' based on Farquhar with low disturbance rate (Disturbance mean interval 40). The size of the point represent the variation of LMA.}
\label{fig:lma_nareaFlSM}}
\end{figure}

\begin{figure}[ht]
\centering
\includegraphics{../figures/gradient_narea_lma_multi_narea_lma2_FvCB.pdf}
\caption{\textbf{Predicted community assembly of leaf mass per area and leaf N per area along a productivity gradient controlled by aridity by 'Plant' based on Farquhar with low disturbance rate (Disturbance mean interval 40). The size of the point represent the variation of LMA.}
\label{fig:lma_nareaFl2SM}}
\end{figure}


\clearpage



\section{Results with FVCB model and a link between N area and leaf turnover rate}


\begin{figure}[ht]
\centering
\includegraphics{../figures/gradient_narea_lma_multi_NvLTR_FvCB_cor.pdf}
\caption{\textbf{LMA vs Leaf N area for model with link between N area and leaf turnover rate with high disturbance rate (Disturbance mean interval 40 years). The color of the points represents the vpd.}
\label{fig:lma_narea_cor_single_NvLTRSM}}
\end{figure}


\begin{figure}[ht]
\centering
\includegraphics{../figures/gradient_narea_lma_multi_narea_lma_NvLTR_FvCB.pdf}
\caption{\textbf{Predicted community assembly of leaf mass per area and leaf N per area along a productivity gradient controlled by aridity by 'Plant' based on Farquhar and a link between N area and leaf turnover rate with low disturbance rate (Disturbance mean interval 40). The size of the point represent the variation of LMA. The run with vpd = 0 failed.}
\label{fig:lma_nareaFlSM}}
\end{figure}

\begin{figure}[ht]
\centering
\includegraphics{../figures/gradient_narea_lma_multi_narea_lma2_NvLTR_FvCB.pdf}
\caption{\textbf{Predicted community assembly of leaf mass per area and leaf N per area along a productivity gradient controlled by aridity by 'Plant' based on Farquhar and a link between N area and leaf turnover rate with low disturbance rate (Disturbance mean interval 40). The size of the point represent the variation of LMA. The run with vpd = 0 failed.}
\label{fig:lma_nareaFl2SM}}
\end{figure}

\bibliographystyle{amnat}
\bibliography{references}
\end{document}
