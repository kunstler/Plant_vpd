\documentclass[11pt,a4paper]{article}
\usepackage[T1]{fontenc}
\usepackage[utf8]{inputenc}
\usepackage{authblk}

\input{common-defs}

\title{How shift in traits composition along climatic gradients emerge from the interplay of climate stress and competition? A theoretical model analysis in forests.}
\author[1]{Georges Kunstler}
\author[2]{Daniel S. Falster}
\author[2]{Richard G. FitzJohn}
\author[2]{Mark Westoby}
\affil[1]{Université Grenoble Alpes, Irstea, UR EMGR, 2 rue de la Papeterie-BP 76, F-38402 St-Martin-d'Hères, France.}
\affil[2]{Department of Biological Sciences, Macquarie University, Sydney, Australia}

\renewcommand\Authands{ and }

\begin{document}
  \maketitle

\begin{abstract}
Understanding changes in vegetation composition across large biogeographic gradients is fundamental, to understand the drivers of both biodiversity and ecosystems functioning at large scale. A long-standing theory is that these changes in forest composition emerge from the interplay of direct climate constraints and biotic interactions, and that biotic processes have a stronger role than abiotic processes in productive climatic conditions. Numerous models have been developed to explore this question, yet few connect with the real climatic gradients and the real traits shaping plants strategies. Here we explore this question with a new model -- \plant\ -- that connects directly with leaf and stem traits and in which competition is driven by competition for light in size-structured meta-populations. More specifically we explore the dynamics emerging for one or multiple traits related to climate stress and competition. Then we compare the model output with field data on forests in Europe and the USA, analysing both the overall shift in traits composition along climatic gradients and the mixture of traits within the communities. This knowledge is crucial to progress our understanding of how geographic boundaries of forest tree species are determined along climate gradients, and how they may move in the future.
\end{abstract}


\end{document}


%% \documentclass[a4paper,11pt]{article}
%% \usepackage[osf]{mathpazo}
%% \usepackage{ms}
%% \usepackage{natbib}
%% \usepackage{graphicx}
%% \usepackage{caption}
%% \usepackage{authblk}
%% \usepackage[labelfont=bf]{caption} % make label for figure bold



%% \date{}
%% \runninghead{}
%% \keywords{}

%% \usepackage{color}


%% \begin{document}

%%  \maketitle


%% \end{document}
