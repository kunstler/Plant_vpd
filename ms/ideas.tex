\documentclass[a4paper,11pt]{article}
\usepackage[osf]{mathpazo}
\usepackage{ms}
\usepackage{natbib}
\usepackage{graphicx}
\usepackage{caption}
\usepackage[labelfont=bf]{caption} % make label for figure bold

% Allow referencing into the supporting information, once that exists.
\IfFileExists{./competition-kernels-sm.tex}{%
  \usepackage{xr}%
  \externaldocument{competition-kernels-sm}}{}

% We will generate all images so they have a width \maxwidth. This means
% that they will get their normal width if they fit onto the page, but
% are scaled down if they would overflow the MAPgins.
\makeatletter
\def\maxwidth{\ifdim\Gin@nat@width>\linewidth\linewidth\else\Gin@nat@width\fi}
\def\maxheight{\ifdim\Gin@nat@height>\textheight\textheight\else\Gin@nat@height\fi}
\makeatother
\setkeys{Gin}{width=\maxwidth,height=\maxheight,keepaspectratio}

\title{Trait gradients}

\author{ Georges Kunstler, Daniel S. Falster, Richard G. FitzJohn}
\date{}
\affiliation{Department of Biological Sciences, Macquarie University,
  Sydney, Australia}
\runninghead{}
\keywords{}

\usepackage{color}

\input{common-defs}

\begin{document}

\mstitleshort
%\mstitlepage
\parindent=1.5em
\addtolength{\parskip}{.3em}

% \begin{abstract}
% Abstract goes here\ldots
% \end{abstract}

\section{Background \& Motivation}

Understanding changes in vegetation composition across large biogeographic gradients is fundamental because vegetation is key to control ecosystem functioning, such as net primary production, and carbon storage. These changes in composition are also fundamental to understand biodiversity, as turnover in species composition explain a large prportion of the global diversity ($\beta$ diversity).

- outcomes of long-term evolutionary adaptation. More than just intrinsic interest, as path towards understanding how vegetation may change as result of environmental change.

Several studies document gradient of key leaf and stem traits with precipitation

- consistent patterns for LMA and WD
- Numerous empirical investigations, but mechanisms remain largely unknown / unexplored.

A promising approach for investigating causes of trait adaptation is via models capturing process of adaptation -- allow you to encode various rules and see how virtual system responds. Many abstract theoretical models, establish some basic principles. For example, Holt, Barton, Taper \& Case. ....

Yet few connect with real gradients and real traits. (What is missing to explain observed empirical patterns). Observed trait gradients are present but weak, majority of variation within sites / climate zones. Need to explain both range (coexistence) and change in mean. There are some optimisation (cost-benefit) models seek to explain changes in traits along gradients  (Sterck?, ask Mark). Have been largely unsuccessful at explaining large-scale gradients because fail to explain (which??) observed patterns (because wrong traits, wrong directions).

In addition, most of \textit{physiological models} have been focused on effect of climate on growth, but much les sis understood in term of efefct on mortality or recruitment. 

Here we use a new model -- \plant\ \citep{Falster-2016} -- that connects directly with leaf and stem traits and simultaneously allows for coexistence within sites and variation across sites.  Driven by competition for light in size-structured metapopulations. Outcomes of competition moderated by environmental factors such as site productivity and disturbance regime.



Key questions

In \plant\ the coexistence between tree is mainly related to difference in successional niche underpinned by LMA. We will explore how the predicted mixture of traits changes along climatic gradient exploring different assumption about how environmental gradients
affect the dynamics. 

1. In the most simple version environmental gradients changes productivity (photosynthesis) and
thus competition for light and selection on LMA (this is already presented in \citet{Falster-2016}).

2. In second version, the environemental gradient affect the productivity ($A{max}$) but also change the trade-off underpinning LMA-LLS in term of its elevation or its slope.

\citet{Wright-2005} reported that the trade-off between LMA and LLS is changing with precipitation with sites with higher precipitation having higher LLS at a given LMA. In contrast, the change in the trade-off between $A_{mass}$ and LMA results in lower $A_{mass}$ at given LMA in sites with higher precipitation. This variation may be related to a change in Leaf N, with species from lower precipitation sites having higher leaf $N_{mass}$ resulting in lower tissue toughness but higher $A_{mass}$ \citep{Wright-2002}. In term in $A_{area}$ this would translate in higher $A_{area}$ at low precipitation sites...  This is a change in optimum $A_{area}$ the $A_{area}$ integrated over the year (as implemented in \plant\ ) is probably higher in site with high precipitation. 

We could explore the effect of changes in elevation of the LLS-LMA trade-off along a precipitation gradient as this is one the stronger pattern in the glopnet data. One issue here is that I think that the results will be strongly influenced by the relative effect of precipitation on $A_{area}$ vs precipitation on the elevation of the trade-off. What kind of slope of precipitation vs $A_{area}$ should we used?

Then the second step would be to explore how LMA interact with a second trait controlling the response to the environmental gardient. This second trait could either affect the response of growth (Amax) or the survival or both. In the litterature there is several traits that have been proposed to act in that way but there is relatively few consus on that. We will start with two traits for which the pattern and mechansims are better understood.
 
2. Aridity gradients affect productivity (photosynthesis) but a
high leaf N allow for an adaptation in term of photosynthesis to low water availabilty but
with a cost (higher leaf N comes at both a respiration cost and higher
leaf turnover cost).

\citet{Wright-2003} explore how N can substitute for water. $A_{area} \propto N_{area} \, g_s$ (with $g_s$ stomatal conductance). One solution would be to assume that the water stress is represented by $1/g_s$ and $N_{area}$ could off-set the effect of water stress on $A_{area}$ (in our case $A_{max}$). The cost of leaf N is in its respiration cost with higher dark leaf respiration per area for higher Narea.

\citet{Prentice-2014} expanded this prediction in term of more details Farquahr model for C3 plant, but this is probably not needed for the \plant\ model. However we need to check if this is providing more numerical estimate of the relation between $N_{area}$ and reponse to water stress. 

3. Frost gradients affect survival but
short vessel size allow for an adaptation to lower temperature in term
of survival but
with a carbon cost (for instance lower vessel size leading to lower
$A_{max}$ \citep{Poorter-2010} but this ref is on RGR).

- For frost tolerance, \citet{Charrier-2013} provides an analysis of physiological processes and 'traits' that may explain variation in tree elevation limits between 11 european tree speciess (including one pine and 10 angiosperms). Their results show that the level of frost cavitation and the ability to refill cavited vessels are driven by the vessel size and the allocation to non-structural carbohydrate (NSC) in winter (helping to improve living tissue survival during frost and increasing the osmotic pressure to refill the vessel). Frost cavitation is in turn strongly correlated to species elevation limit. Thus small vessel size and high allocation to NSC could favor survival in frost conditions. The cost of small vessel size has been shown to be a smaller growth rate \citep{Poorter-2010} (include that into $A_{max}$?) and allocation to NSC have a direct carbon cost.


4. Then there is a growing litterature on traits controlling survival in drought, but our understanding of the traits controlling it is still limited, so including that in the model is not so easy.

- Water stress mortality can be caused either by carbon starvation or hydraulic failure \citep{McDowell-2008,McDowell-2011,Skelton-2015}. Isohydric plants decrease quickly their $g_s$ (stomatal conductance) during water stress to avoid caviation but at the expense of carbon uptake. In contrast, anisohydric plants have high $g_s$ during water stress to maintain carbon uptake but have greater risk of cavitation and the cost of constructing xylem resisting high water deficit is high. There is a tradeoff between stomatal regulation and tolerance to low water potential in term of caviation. But it is unclear which 'easy to measure' traits could underpin this tradeoff between this two different axis of tolerance to water stress.


\section{Review: Empirical patterns of trait variation across abiotic
gradients}


\subsection{Leaf mass per area}

\begin{itemize}
\item \citet{Niinemets-2001}: compared LMA for 558 broad-leaved and 39 needle-leaved shrubs and trees from 182 geographical locations distributed. LMA decreasing with rainfall (in driest month). Also increasing with irradiance. Thickness and LMA also increasing with MAT.

\item \citet{Wright-2002}: compared relationships between LMA and leaf lifespan at two low and two high rainfall sites. Low rainfall sites tended to have slightly high LMA and also lower LL at given LMA, i.e. intercept of relationship decreased. Species from low rainfall sites were also easier to cut (work to shear) at given LMA, so the interpretation was that lower LL for given LMA was because leaves consisted of soft internal tissues (e.g. more mesophyll, high leaf N).

\item \citet{Wright-2004}: Data 2,370 species from 163 sites. LMA increases with MAT and decrease with MAP (figure 3).

\item \citet{Wright-2005}: At hotter, drier and higher irradiance sites mean LMA was higher; At drier sites average LL was shorter at a given LMA (confirming patterns from Wright 2002); at hotter sites the increase in LL was less for a given increase in LMA (LL-LMA slope became less positive)

\item \citet{Onoda-2011}: data for 2819 species from 90 sites worldwide. LMA tends to be higher on drier sites.

\item \citet{Moles-2014}: data for 10792 species from 722 sites. LMA decreases with MAP, increases with MAT.

\item \citet{Simova-2015}: across entire Nth American continent, LMA decreasing with MAT and MAP, increasing with degree of seasonality. No significant change in variance. Note patterns with temperature in this study is opposite to all the above results. Also decreases with MAT but increases with degree of seasonality -- is this conflicting?

\item \citet{Laughlin-2012}: Data for nine tree species in the south-west USA across 196 plots,between 2200 and 3600 m altitude, corresponding to a 10 degC range in MAT. LMA decreased with MAT.

\item \citet{Fortunel-2014}: higher LMA  in white sand forest than in flooded forest or terra firme forest (with less water stress). However, variation within site is larger than between sites.

\item \citet{vanOmmenKloeke-2012}: looked at leaf lifespan for 189 deciduous and 506 evergreen species across 83 study locations. The LLS of deciduous and evergreen species showed opposite responses to temperature: deciduous LLS increases while evergreen LLS decreases. Pattern with MAP was weak. Note, similar result already reported by \citet{Wright-2005}.

\item \citet{Douma-2011}: data from 156 plots from natural ecosystems throughout The Netherlands.  Decrease in LMA with nutrient availability.

\item \citet{VanBodegom-2014}: LMA increase with PET and decrease with number of frost days and the number of wet days (precip > PET). But stronger effect of Nmin (net nitrogen mineralization rate).

\item \citet{Cornwell-2009}: LMA decrease with soil water availability and increase with elevation.

\item \citet{Maire-2015}: LMA decrease with Tmax and increase number of frost days.
\end{itemize}

\subsection{Wood density}

\begin{itemize}
  \item \citet{Fortunel-2014}: higher wood density in white sand forest than in flooded forest or terra firme forest (with less water stress). However, variation within site is larger than between sites.

\item \citet{Swenson-2010}: Mean wood density of US forest inventory plots increase with precipitation and temperature. But variance of wood density decrease with Tmax.

\item \citet{VanBodegom-2014}: Wood density decrease with number of frost days.

\item \citet{Laughlin-2012}: Data for nine tree species in the south-west USA across 196 plots,between 2200 and 3600 m altitude, corresponding to a 10 degC range in MAT. Concave-shaped relationship with MAT.

\item \citet{Simova-2014}: across entire Nth American continent. Weak non-significant pattern of WD increasing with MAT, decreasing with MAP, and variance decreasing with MAP (lower variance in high rainfall).

\item \citet{Chave-2006}: WD for 2456 tree species from Central and South America. Wood density varied over more than one order of magnitude across species, with an overall mean of 0.645 g/cm3. Our geographical analysis showed significant decreases in wood density with increasing altitude and significant differences among low-altitude geographical regions: wet forests of Central America and western Amazonia have significantly lower mean wood density than dry forests of Central and South America, eastern and central Amazonian forests, and the Atlantic forests of Brazil; and eastern Amazonian forests have lower wood densities than the dry forests and the Atlantic forest.

\item \citet{Chave-2009}: some weak evidence that WD max be more variable in tropic than temperate zone. A negative relationship between wood density and soil fertility appears widespread (Baker et al. 2004; ter Steege et al. 2006), as does a positive but weaker relationship between wood density and temperature. Little relationship with rainfall (cite Swenson). Yet Fig 6 suggests can predict 50\% of variance in mean WD across nth and Sth America from mean annual temperature, annual precipitation, precipitation seasonality and temperature seasonality.

\item \citet{Kooyman-2010}: plot means of WD in Australian rainforests tends to increase with soil texture and decrease with soil depth.

\item \citet{Preston-2006}: Wood density decrease soil water availability, vessel area increase with water availability.

\item \citet{Cornwell-2009}: Wood density decrease with soil water availability and increase with elevation. Lumen fraction fraction and vessel area increase with soil water availability.
\end{itemize}

\citet{Stahl-2014}: Wood density increase with MAT but nore relation with MAP.

\subsection{Leaf nitrogen per area}

\begin{itemize}
\item \citet{Wright-2005}: Data 2,370 species from 163 sites. At hotter, drier and higher irradiance sites, leaf N per area was higher. Pattern strongest with MAP and irradiance (also VPD and PET), but only weak change with MAT.

\item \citet{Swenson-2010}: Mean leaf nitrogen of US forest inventory plot decrease with precipitation and temperature. And variance of variance of leaf nitrogen decrease with Tmax.

\item \citet{Moles-2014}: data for 10792 species from 722 sites. Narea decrease with increasing MAT. Non-sig with MAP.

\item \citet{Maire-2015}: Narea decrease with Precip/PET and temperature range (Tmax - Tmin).

\end{itemize}

\subsection{Leaf nitrogen per mass}

\begin{itemize}
\item \citet{Reich-2004}: Nmass highest at average temperature (lower in tropical forest and tundra).

\item \citet{Moles-2014}: data for 10792 species from 722 sites. Nmass decrease with increasing MAT.

\item \citet{Ordonez-2009}: Nmass increase with MAT.
\end{itemize}

\subsection{Summary}

There appears to be a clear pattern of decreasing LMA and decreasing WD with increasing water availability (measured either as mean annual precip, or some other measure of wetness). This relationship was found in all studies and areas, with but one exception: \citet{Swenson-2010} found an increase in WD with MAP across the north America. However, it is important to note that Swenson's study infers WD from species distribution maps, estimates are likely to be more uncertain than where known occurrences are used. Further, using similar dataset, \citet{Stahl-2015} found no relationship between MAP and WD.

\section{Conceptual overview of analysis}

\subsection{Diversification in successional strategy}

Requires demographic trade-off between growth in high-light and shade-tolerance.

Possible via trade-offs in either LMA or wood density,


1. Change in productivity
  - high LMA as adaptation to both log light and low water
2. Change the relationship LMA and turnover
  - slope or elevation
3. Introduce additional trait. 3 possible outcomes
  - compensates for lower water, leaf N per area
    - make LMA shift less than would have been otherwise
    - instead differentiate species in N (with flow on effect to shade tolerance) and also in LL
  - ameliorates risk of mortality
    - (do we know which trait changes embolism risk)
      - psi50: don't know which traits
      - vessel size:
    - assume has some carbon cost, otherwise not beneficial
    - diverts carbon away from growth, so amplifies change in LMA
  - height at maturation
    - will decrease difference in LMA

Tasks

1. get evolutionary assembly working again
  - basic patterns for lma with site productivity
  - Sensitivity of LMA to different factors (get in right range)
2. implement/improve leaf Nitrogen physiology
  - link for water?
3. Define new risk amelioration trait and effect on mortality
4. Review treatment of water in DGVMs


\subsection{Stress tolerance trait}





\clearpage

\bibliographystyle{amnat}
\bibliography{references}
\end{document}
