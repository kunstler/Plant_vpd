\documentclass[a4paper,11pt]{article}
\usepackage[osf]{mathpazo}
\usepackage{ms}
\usepackage{natbib}
\usepackage{graphicx}
\usepackage{caption}
\usepackage[labelfont=bf]{caption} % make label for figure bold

% Allow referencing into the supporting information, once that exists.
\IfFileExists{./competition-kernels-sm.tex}{%
  \usepackage{xr}%
  \externaldocument{competition-kernels-sm}}{}

% We will generate all images so they have a width \maxwidth. This means
% that they will get their normal width if they fit onto the page, but
% are scaled down if they would overflow the MAPgins.
\makeatletter
\def\maxwidth{\ifdim\Gin@nat@width>\linewidth\linewidth\else\Gin@nat@width\fi}
\def\maxheight{\ifdim\Gin@nat@height>\textheight\textheight\else\Gin@nat@height\fi}
\makeatother
\setkeys{Gin}{width=\maxwidth,height=\maxheight,keepaspectratio}

\title{Trait gradients}

\author{ Georges Kunstler, Daniel S. Falster, Richard G. FitzJohn}
\date{}
\affiliation{Department of Biological Sciences, Macquarie University,
  Sydney, Australia}
\runninghead{}
\keywords{}

\usepackage{color}

\newcommand{\ud}{\ensuremath{\mathrm{d}}}
\newcommand{\sign}{\mathop{\mathrm{sign}}\nolimits}
\newcommand{\Rstar}{\ensuremath{R^*}}
\newcommand{\plant}{{\tt plant}}
\newcommand{\hmat}{\ensuremath{h_{\text{mat}}}}
\newcommand{\TODO}{{\color{red}\sc todo}}


\begin{document}

\mstitleshort
%\mstitlepage
\parindent=1.5em
\addtolength{\parskip}{.3em}

% \begin{abstract}
% Abstract goes here\ldots
% \end{abstract}

\section{Background \& Motivation}

Understanding changes in vegetation composition across large biogeographic gradients is fundamental because vegetation is key to control ecosystem functioning, such as net primary production, and carbon storage. These changes in composition are also fundamental to understand biodiversity, as turnover in species composition explain a large proportion of the global diversity ($\beta$ diversity). In addition,  This pattern are the outcomes of long-term evolutionary adaptation. More than just intrinsic interest, as path towards understanding how vegetation may change as result of environmental change.

Several studies document gradient of key leaf and stem traits with precipitation

\begin{itemize}
\item consistent patterns for LMA and WD
\item Numerous empirical investigations, but mechanisms remain largely unknown / unexplored.
\end{itemize}

A promising approach for investigating causes of trait adaptation is via models capturing process of adaptation -- allow you to encode various rules and see how virtual system responds. Many abstract theoretical models, establish some basic principles. For example, Holt, Barton, Taper \& Case. ....
Most of the abstract theoretical are based on strong assumption about how traits are link to the competition and the response to the environmental gradient (for instance a traits describing the local adaptation to the local environmental conditions also drives the competition through a normal competition kernel or function). Very few model start from the mechanisms of competition for resources and the physiology of the response to the environmental gradient to predict how local adaptation and competition will shape the community assembly.

Yet few connect with real gradients and real traits.    (What is missing to explain observed empirical patterns). Observed trait gradients are present but weak, majority of variation within sites / climate zones. Need to explain both range (coexistence) and change in mean. There are some optimisation (cost-benefit) models seek to explain changes in traits along gradients  (Sterck?, ask Mark). Have been largely unsuccessful at explaining large-scale gradients because fail to explain (which??) observed patterns (because wrong traits, wrong directions).
% We need to organise more clearly this section as optimisation model
% can also be viewed as adaptation model. The two mains points would
% be (i) model and their limitations and (ii) the observed gradients
% with higher variance within site than between sites and their lack
% of connection with model. The output would be to show that model need
% to be based on key physiological mechanism to allow connection with
% the field and that to deal with the large within site variance
% model need to include interplay between mechanisms allowing
% coexistence and abiotic effect. It may be also interesting to have
% few lines one abiotic filtering debate.
In addition, most of \textit{physiological models} have been focused on effect of climate on growth, but much less is understood in term of effect on mortality or recruitment.

Here we use a new model -- \plant\ \citep{Falster-2016} -- that connects directly with leaf and stem traits and simultaneously allows for coexistence within sites and variation across sites.  Driven by competition for light in size-structured meta-populations. Outcomes of competition moderated by environmental factors such as site productivity and disturbance regime.

\section{Key questions}

In \plant\ the coexistence between tree is mainly related to difference in successional niche underpinned by LMA through competition for light. We will explore how the predicted mixture of traits changes along climatic gradient exploring different assumption about how environmental gradients
affect the dynamics and competition for light. There is several
potential mechanisms through which the environmental gradient could
affect the community assembly. First the gradient can simply affect
the productivity with consequence on the process of light
competition. Secondly, the environmental gradient can affect the
trade-off underpinning the strategy of light use (so changing the
trade-off between LMA and LLS). Then the trait controlling competition
for light may interact with other traits controlling the effect of the
gradients on either growth as in the two previous points, or through
mortality. Effects on growth or survival may have different effects as
they may interact in different way with competition for light). Our knowledge of the traits and potential mechanisms that
may determine species response to different abiotic gradient is still
limited. Here we will explore two traits for which our understanding
is more advanced Leaf N along precipitation gradients affecting growth
and frost and vessel size (allocation to reserve as well ??) affecting
survival.

\section{Conceptual overview of analysis}

\subsection{Diversification in successional strategy and environmental effect}

Coexistence across a successional gradient requires demographic trade-off between growth in high-light and shade-tolerance. Possible via trade-offs in either LMA or wood density (height at maturation - will decrease difference in LMA). Then the question is how environmental gradients affect this fundamental process of coexistence in forest and we will explore different assumption about how environmental gradients
affect the dynamics.



1. In the most simple version environmental gradients changes productivity (photosynthesis) and
thus competition for light and selection on LMA (this is already presented in \citet{Falster-2016}) (high LMA as adaptation to both log light and low productivity).

2. In second version, the environmental gradient affect the productivity ($A{max}$) but also change the trade-off underpinning LMA-LLS in term of its elevation or its slope.

\begin{itemize}
\item \citet{Wright-2005} reported that the trade-off between LMA and Leaf Turnover Rate, LTR ($1/LLS$), is changing with aridity with higher aridity sites having higher LTR at a given LMA and a shallower slope. There is also a shift in the link $A_{mass}$ and LMA (with higher in dry sites), probably because leaf N, but in a first step we will ignore that as the current version consider $A_{area}$ constant and the photosynthesis integrated over the year (as implemented in \plant\ ) is probably lower in dry sites. So we will keep the link between $a_{p1}$ and stress gradient as above. Note that \citet{Sakschewski-2015} implement a link between LMA and $v_{cmax_{area}}$.

%% In contrast, the change in the trade-off between $A_{mass}$ and LMA results in lower $A_{mass}$ at given LMA in sites with higher precipitation. This variation may be related to a change in Leaf N, with species from lower precipitation sites having higher leaf $N_{mass}$ resulting in lower tissue toughness but higher $A_{mass}$ \citep{Wright-2002}. In term in $A_{area}$ this would translate in higher $A_{area}$ at low precipitation sites...  This is a change in optimum $A_{area}$ the $A_{area}$ integrated over the year (as implemented in \plant\ ) is probably higher in site with high precipitation.

\item We could explore the effect of changes in elevation of the LLS-LMA trade-off along a precipitation gradient as this is one the stronger pattern in the Glopnet data. One issue here is that I think that the results will be strongly influenced by the relative effect of precipitation on $A_{area}$ vs precipitation on the elevation of the trade-off. What kind of slope of precipitation vs $A_{area}$ should we used?

\end{itemize}

Then the second step would be to explore how LMA interact with a second trait controlling the response to the environmental gradient. This second trait could either affect the response of growth ($A_{max}$) or the survival or both. In the literature there is several traits that have been proposed to act in that way but there is relatively few consensus on that. We will start with two traits for which the pattern and mechanisms are better understood.

3. Aridity gradients affect productivity (photosynthesis) but a
high leaf N allow for an adaptation in term of photosynthesis to low water availability but with a cost (higher leaf N comes at both a respiration cost and higher leaf turnover cost). This may results in LMA shifting less than would have been otherwise, but  instead differentiate species in N (with flow on effect to shade tolerance).


\begin{itemize}
\item \citet{Wright-2003} explore how N can substitute for water. $A_{area} \propto N_{area} \, g_s$ (with $g_s$ stomatal conductance). One solution would be to assume that the water stress is represented by $1/g_s$ and $N_{area}$ could off-set the effect of water stress on $A_{area}$ (in our case $A_{max}$). The cost of leaf N is in its respiration cost with higher dark leaf respiration per area for higher $N_{area}$.

\item \citet{Prentice-2014} expanded this prediction in term of more details Farquhar model for C3 plant, but this is probably not needed for the \plant\ model. However we need to check if this is providing more numerical estimate of the relation between $N_{area}$ and response to water stress.
\end{itemize}

4. Frost gradients affect survival but
short vessel size allow for an adaptation to lower temperature in term
of survival but
with a carbon cost (for instance lower vessel size leading to lower
$A_{max}$ \citep{Poorter-2010}, we know that vessel diameter scale with scale with sapwood area conductivity and $A_{max}$ \citep{Chen-2009,Choat-2011}).


\begin{itemize}
\item  For frost tolerance, \citet{Charrier-2013} provides an analysis of physiological processes and 'traits' that may explain variation in tree elevation limits between 11 European tree species (including one pine and 10 angiosperms). Their results show that the level of frost cavitation and the ability to refill cavitated vessels are driven by the vessel size and the allocation to non-structural carbohydrate (NSC) in winter (helping to improve living tissue survival during frost and increasing the osmotic pressure to refill the vessel). Frost cavitation is in turn strongly correlated to species elevation limit. Thus small vessel size and high allocation to NSC could favour survival in frost conditions. The cost of small vessel size has been shown to be a smaller growth rate \citep{Poorter-2010} (include that into $A_{max}$?) and allocation to NSC have a direct carbon cost. Diverting carbon from growth in to NSC may have the effect of amplifying the changes in LMA.
\item TO DO need to read \citet{Markesteijn-2011} for potential coordination between vessel size and shade-tolerance as well.
\end{itemize}

5. Then there is a growing literature on traits controlling survival in drought, but our understanding of the traits controlling it is still limited, so including that in the model is not so easy.

\begin{itemize}
\item Water stress mortality can be caused either by carbon starvation or hydraulic failure \citep{McDowell-2008,McDowell-2011,Skelton-2015}. Isohydric plants decrease quickly their $g_s$ (stomatal conductance) during water stress to avoid cavitation but at the expense of carbon uptake. In contrast, anisohydric plants have high $g_s$ during water stress to maintain carbon uptake but have greater risk of cavitation and the cost of constructing xylem resisting high water deficit is high. There is a trade-off between stomatal regulation and tolerance to low water potential in term of cavitation (measured by $\psi_{50}$). But it is unclear which 'easy to measure' traits could underpin this trade-off between this two different axis of tolerance to water stress. Even for $\psi_{50}$ alone we don't know which traits is relevant as vessel size is not a good proxy \citep{Maherali-2004}.
\end{itemize}





\clearpage

\bibliographystyle{amnat}
\bibliography{references}
\end{document}

%%  LocalWords:  LMA WD optimisation Falster successional LLS et
%%  LocalWords:  Glopnet stomatal Farquhar Poorter RGR Charrier
%%  LocalWords:  cavitation cavitated Skelton Isohydric Onoda al
%%  LocalWords:  anisohydric Niinemets Simova Laughlin Fortunel
%%  LocalWords:  Douma VanBodegom Kooyman rainforests
