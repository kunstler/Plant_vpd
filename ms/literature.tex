\documentclass[a4paper,11pt]{article}
\usepackage[osf]{mathpazo}
\usepackage{ms}
\usepackage{natbib}
\usepackage{graphicx}
\usepackage{caption}
\usepackage[labelfont=bf]{caption} % make label for figure bold

% We will generate all images so they have a width \maxwidth. This means
% that they will get their normal width if they fit onto the page, but
% are scaled down if they would overflow the MAPgins.
\makeatletter
\def\maxwidth{\ifdim\Gin@nat@width>\linewidth\linewidth\else\Gin@nat@width\fi}
\def\maxheight{\ifdim\Gin@nat@height>\textheight\textheight\else\Gin@nat@height\fi}
\makeatother
\setkeys{Gin}{width=\maxwidth,height=\maxheight,keepaspectratio}

\title{Literature about trait gradients}

\author{ Georges Kunstler, Daniel S. Falster}
\date{}
\affiliation{}
\runninghead{}
\keywords{}

\usepackage{color}

\begin{document}

\mstitleshort
%\mstitlepage
\parindent=1.5em
\addtolength{\parskip}{.3em}

% \begin{abstract}
% Abstract goes here\ldots
% \end{abstract}



\section{Empirical patterns of trait variation across abiotic
gradients}


\subsection{Leaf mass per area}

\begin{itemize}
\item \citet{Niinemets-2001}: compared LMA for 558 broad-leaved and 39 needle-leaved shrubs and trees from 182 geographical locations distributed. LMA decreasing with rainfall (in driest month). Also increasing with irradiance. Thickness and LMA also increasing with MAT.

\item \citet{Wright-2002}: compared relationships between LMA and leaf lifespan at two low and two high rainfall sites. Low rainfall sites tended to have slightly high LMA and also lower LL at given LMA, i.e. intercept of relationship decreased. Species from low rainfall sites were also easier to cut (work to shear) at given LMA, so the interpretation was that lower LL for given LMA was because leaves consisted of soft internal tissues (e.g. more mesophyll, high leaf N).

\item \citet{Wright-2004}: Data 2,370 species from 163 sites. LMA increases with MAT and decrease with MAP (figure 3).

\item \citet{Wright-2005}: At hotter, drier and higher irradiance sites mean LMA was higher; At drier sites average LL was shorter at a given LMA (confirming patterns from Wright 2002); at hotter sites the increase in LL was less for a given increase in LMA (LL-LMA slope became less positive)

\item \citet{Onoda-2011}: data for 2819 species from 90 sites worldwide. LMA tends to be higher on drier sites.

\item \citet{Moles-2014}: data for 10792 species from 722 sites. LMA decreases with MAP, increases with MAT.

\item \citet{Simova-2015}: across entire North American continent, LMA decreasing with MAT and MAP, increasing with degree of seasonality. No significant change in variance. Note patterns with temperature in this study is opposite to all the above results. Also decreases with MAT but increases with degree of seasonality -- is this conflicting?,.  

\item \citet{Laughlin-2012}: Data for nine tree species in the south-west USA across 196 plots,between 2200 and 3600 m altitude, corresponding to a 10 degC range in MAT. LMA decreased with MAT.

\item \citet{Fortunel-2014}: higher LMA  in white sand forest than in flooded forest or terra firme forest (with less water stress). However, variation within site is larger than between sites.

\item \citet{vanOmmenKloeke-2012}: looked at leaf lifespan for 189 deciduous and 506 evergreen species across 83 study locations. The LLS of deciduous and evergreen species showed opposite responses to temperature: deciduous LLS increases while evergreen LLS decreases. Pattern with MAP was weak. Note, similar result already reported by \citet{Wright-2005}.

\item \citet{Douma-2011}: data from 156 plots from natural ecosystems throughout The Netherlands.  Decrease in LMA with nutrient availability.

\item \citet{VanBodegom-2014}: LMA increase with PET and decrease with number of frost days and the number of wet days (precip > PET). But stronger effect of Nmin (net nitrogen mineralization rate).

\item \citet{Cornwell-2009}: LMA decrease with soil water availability and increase with elevation.

\item \citet{Maire-2015}: LMA decrease with Tmax and increase number of frost days.
\end{itemize}

\subsection{Wood density}

\begin{itemize}
  \item \citet{Fortunel-2014}: higher wood density in white sand forest than in flooded forest or terra firme forest (with less water stress). However, variation within site is larger than between sites.

\item \citet{Swenson-2010}: Mean wood density of US forest inventory plots increase with precipitation and temperature. But variance of wood density decrease with Tmax.

\item \citet{VanBodegom-2014}: Wood density decrease with number of frost days.

\item \citet{Laughlin-2012}: Data for nine tree species in the south-west USA across 196 plots,between 2200 and 3600 m altitude, corresponding to a 10 degC range in MAT. Concave-shaped relationship with MAT.

\item \citet{Simova-2015}: across entire Nth American continent. Weak non-significant pattern of WD increasing with MAT, decreasing with MAP, and variance decreasing with MAP (lower variance in high rainfall).

\item \citet{Chave-2006}: WD for 2456 tree species from Central and South America. Wood density varied over more than one order of magnitude across species, with an overall mean of 0.645 g/cm3. Our geographical analysis showed significant decreases in wood density with increasing altitude and significant differences among low-altitude geographical regions: wet forests of Central America and western Amazonia have significantly lower mean wood density than dry forests of Central and South America, eastern and central Amazonian forests, and the Atlantic forests of Brazil; and eastern Amazonian forests have lower wood densities than the dry forests and the Atlantic forest.

\item \citet{Chave-2009}: some weak evidence that WD max be more variable in tropic than temperate zone. A negative relationship between wood density and soil fertility appears widespread (Baker et al. 2004; ter Steege et al. 2006), as does a positive but weaker relationship between wood density and temperature. Little relationship with rainfall (cite Swenson). Yet Fig 6 suggests can predict 50\% of variance in mean WD across nth and Sth America from mean annual temperature, annual precipitation, precipitation seasonality and temperature seasonality.

\item \citet{Kooyman-2010}: plot means of WD in Australian rainforests tends to increase with soil texture and decrease with soil depth.

\item \citet{Preston-2006}: Wood density decrease soil water availability, vessel area increase with water availability.

\item \citet{Cornwell-2009}: Wood density decrease with soil water availability and increase with elevation. Lumen fraction fraction and vessel area increase with soil water availability.
\end{itemize}

\citet{Stahl-2014}: Wood density increase with MAT but no relation with MAP.

\subsection{Leaf nitrogen per area}

\begin{itemize}
\item \citet{Wright-2005}: Data 2,370 species from 163 sites. At hotter, drier and higher irradiance sites, leaf N per area was higher. Pattern strongest with MAP and irradiance (also VPD and PET), but only weak change with MAT.

\item \citet{Swenson-2010}: Mean leaf nitrogen of US forest inventory plot decrease with precipitation and temperature. And variance of variance of leaf nitrogen decrease with Tmax.

\item \citet{Moles-2014}: data for 10792 species from 722 sites. Narea decrease with increasing MAT. Non-sig with MAP.

\item \citet{Maire-2015}: Narea decrease with Precip/PET and temperature range (Tmax - Tmin).

\end{itemize}

\subsection{Leaf nitrogen per mass}

\begin{itemize}
\item \citet{Reich-2004}: Nmass highest at average temperature (lower in tropical forest and tundra).

\item \citet{Moles-2014}: data for 10792 species from 722 sites. Nmass decrease with increasing MAT.

\item \citet{Ordonez-2009}: Nmass increase with MAT.
\end{itemize}

\subsection{Vessel size}

Central for this traits is the idea of a trade-off between safety
\textit{vs.} efficiency of the xylem vessel
networks. \citet{Gleason-2016} however shows the existence of numerous
species with low safety and efficiency. Vessel diamter
is related to the conductivity through. The link between vessel size
and drought caviation risk is less clear, probably because of the
importance of intervessel pits in controlling caviation in some
species (in contrast with freeze–thaw embolism which is more directly
related to vessel).  
\citet{Pfautsch-2016} pattern of decrease of mean hydraulically
weighted vessel diameter with increasing
aridity in Australia independent of tree height, in contrast with
\citet{Olson-2013,Olson-2014} which found little effect of climate after
controlling for tree height ($R^2 = 0.06$).

\citet{Markesteijn-2011} No direct measure of vessel size but report a
a trade-off between hydraulic conductivity and both shade tolerance
and drought tolerance (in oposition to what is generally expected).

Other caracteristic of the vascular system are also crucial, pit
membrane anatomy (such as Intervessel pit membrane thickness), Hydraulic integration, vessel agregation, ...

\subsection{Leaf and wood tolerance to water stress P50}

P50 of wood (12 or 88 as well)

P50leaf \citet{Blackman-2012,Blackman-2014}  and leaf water potential at turgor loss point $\pi_{tlp}$.
TODO

\subsection{Leaf Photosynthesis response to water or to temperature}

TODO

\subsection{LT50 Frost tolerance of leaf or wood}

TODO

\subsection{Leaf area to sapwood ratio}


\subsection{Summary}

There appears to be a clear pattern of decreasing LMA and decreasing WD with increasing water availability (measured either as mean annual precip, or some other measure of wetness). This relationship was found in all studies and areas, with but one exception: \citet{Swenson-2010} found an increase in WD with MAP across the north America. However, it is important to note that Swenson's study infers WD from species distribution maps, estimates are likely to be more uncertain than where known occurrences are used. Further, using similar data set, \citet{Stahl-2015} found no relationship between MAP and WD.


\section{Review: Evolution along environmental gradients}

There are numerous studies from the evolutionary literature investigating adaptation and character displacement along environmental gradients. All start with the assumption that there is an gradual cline in optimum trait value along environmental gradient, and then given this starting point investigate how competitive interactions affect process of adaptation along the gradient.

Range limits:

\begin{itemize}
\item \citet{Kirkpatrick-1997}: Haldane argued that species' ranges could be set intraspecifically when gene flow from a species' populous center over-whelms local adaptation at the periphery. Kirkpatrick and Barton modeled Haldane's process with a quantitative genetic model that combines density-dependent local population growth with dispersal and gene flow across a linear environmental gradient in optimum phenotype.  One of 3 outcomes is possible: the species will become extinct, expand to fill all of the available habitat, or be confined to a limited range in which it is sufficiently adapted to allow population growth. When the environment changes rap-idly in space, increased migration inhibits local adaptation and so decreases the species' total population size. Gene flow can cause enough maladaptation that the peripheral half of a species' range acts as a demographic sink.
\item \citet{Case-2000}:  Extended the Kirkpatrick and Barton model to include interspecific competition and the frequency-dependent selection that it generates, as well as stabilizing selection on a quantitative character. Reproduces the Kirkpatrick and Barton single-species result that limited ranges can be produced with sufficiently steep environmental gradients and strong dispersal. Also finds that interspecific competition can interact with environmental gradients and gene flow to generate limited ranges with much less extreme gradient and dispersal parameters than in the single-species case. Stable range limits can be seen as ultimately arising from insufficient relevant genetic variability for natural selection at range boundaries and/or gene flow from the interior of the range swamping out local adaptation at the range boundaries
\item \citet{Goldberg-2006}: Relate results to type of competition. Present four spatial models of character displacement in quantitative traits affecting competition and hybridization between the species. Our models highlight the connections between range limits and character displacement in continuous space. We conclude that the classic pattern (greater phenotypic difference between species in sympatry than allopatry) is less likely to occur for a trait affecting resource acquisition than for a trait affecting mate choice.
\item \citet{Bridle-2007}: Review on factors limiting range edges
\end{itemize}

Clustering along gradient:

\begin{itemize}
\item \citet{Doebeli-2003}: Show that along an environmental gradient, evolutionary branching can occur much more easily than in non-spatial models. This facilitation is most pronounced for gradients of intermediate slope. Moreover, spatial evolutionary branching readily generates patterns of spatial segregation and abutment between the emerging species.
\item \citet{Polechova-2005}: Questions findings of \citet{Doebeli-2003}, arguing that results solely due to boundary effects.  Argued that a gradual environmental cline will lead to gradual variation in a quantitative trait.
\item \citet{Leimar-2008}: Revisits question of clustering along gradient. Extends  \citet{Doebeli-2003} into a continuous model based on reaction-diffusion set-up. Finds that clustering not possible with competition Gaussian kernel, but is possible with other non-Gaussian shapes (see \citet{Leimar-2013} for further extensions).

\end{itemize}


\section{Theoretical set-ups}

\subsection{Discrete populations}

One approach is to model a discrete number of populations linked by dispersal. The approach below follows \citet{Pontarp-2015}, who looked at evolution across two patches.

Let $x_i$ be the trait of species $i$ and $\mathbf{x} = [x_1,\ldots, x_i]$ be the vector of trait values for all species in the system. Now denote $n_{i,A}, n_{i,B}$ to be the density of species $i$ in two patches (A and B), and the vector $\mathbf{n} = [n_{1,A}, n_{1,B}, \ldots, n_{i,A}, n_{i,B}]$ to be the density of all species across all patches.

We denote $R_j(x_i, \mathbf{n})$ to be the fitness of individuals with trait $x_i$ in patch $j$. The populations dynamics for the density of species $i$ are then given by
$$n_{i,A}(t+1) = (1-m) \,(1+R_A(x_i, \mathbf{n})) \, n_{i,A}(t) + v\, n_{i,B}(t),$$
$$n_{i,B}(t+1) = m  \, (1+R_A(x_i, \mathbf{n}))\, n_{i,A}(t) + (1-m) \,  (1+R_B(x_i, \mathbf{n})) \, n_{i,B}(t),$$
where $m$ is the probability of dispersing from one patch to the other. Using matrix algebra the dynamics of $\mathbf{n_i} = [n_{i,A}, n_{i,B}]$ can be written as
$$\mathbf{n_i}(t+1)=\mathbf{M}(x_i, \mathbf{n}) \mathbf{n_i},$$
where
$$ \mathbf{M}(x_i, \mathbf{n}) =  \left[ \begin{array}{cc}
(1-m) \,(1+R_A(x_i, \mathbf{n})) & m (1+R_B(x_i, \mathbf{n})) \\
m \,(1+R_A(x_i, \mathbf{n})) & (1-m) (1+R_B(x_i, \mathbf{n})) \\
\end{array} \right].$$

The per-capita growth rate of the $x_i$ individuals across the meta-population is then given by the dominant eigen value of $\mathbf{M}(x_i, \mathbf{n})$. The equilibrium population density is given by a vector $\mathbf{n^*}$.

A nice feature of this set-up is that when $m=0$ (patches completely decoupled), $\mathbf{M}(x_i, \mathbf{n})$ becomes a diagonal matrix with eigen-value given by the average of fitness across the two patches.


\subsection{Reaction-diffusion models}

Another approach is to model the continuous change in trait values across the gradient. below I describe two different approaches. note that neither of these go so far as to model the full adaptive dynamics --  this would be difficult because the density distributions for each phenotype are continuous functions.

\citet{Case-2000} models the continuous distribution of traits for two species. The system is designed to allow gradual changes in phenotype for each species as well as range boundaries and species turnover.  The core elements of the setup (too complicated to repeat here) are as follows:
\begin{itemize}
  \item Population dynamics based on Lotka-Volterra equations
  \item Competition at given spatial location based on trait-similarity
  \item Optimum trait value which is gradually changing along the gradient
  \item For each species model the distribution $N_i(x,t)$: the density of phenotype $i$ at spatial location $x$ and at time $t$.
  \item Changes in phenotype distributions at location $x$ given by standard approach from Lande 196, including heritability term
  \item Movements and gene flow addded via a diffusion term $D\partial ^2 N_i(x,t)/ \partial x^2$
\end{itemize}
From an initial distribution of phenotypes for one or more species, it is possible to simulate the expansion and gradual adaptation of the population along the gradient.


\cite{Leimar-2008} likewise model adaptation along an environmental gradient using a continuous reaction diffusion. The model is presented as a generalisation of \citet{Doebeli-2003}, which is an individual based model. The key features of their set-up are as follows:
\begin{itemize}
  \item Population dynamics based on Lotka-Volterra equations
  \item Only tracks trait distribution, i.e. density $N(u,x)$ of trait values $u$ at location $x$. Does not worry about designating these into species. Therefore says nothing about species turnover and range edges
  \item Competition at given spatial location based on trait-similarity. In addition there is also competition along the gradient. Trait and distance based competition occurs via kernels which can have varying shapes
  \item Optimum trait value which is gradually changing along the gradient
  \item Movements and gene flow added via explicit movement and mutation kernels
  \item Adds Allee effect (increase in fitness at low densities).
\end{itemize}


\clearpage

\bibliographystyle{amnat}
\bibliography{references}
\end{document}

%%  LocalWords:  LMA WD optimisation Falster successional LLS et
%%  LocalWords:  Glopnet stomatal Farquhar Poorter RGR Charrier
%%  LocalWords:  cavitation cavitated Skelton Isohydric Onoda al
%%  LocalWords:  anisohydric Niinemets Simova Laughlin Fortunel
%%  LocalWords:  Douma VanBodegom Kooyman rainforests
