\documentclass[a4paper,11pt]{article}
\usepackage[osf]{mathpazo}
\usepackage{ms}
\usepackage{natbib}
\usepackage{graphicx}
\usepackage{caption}
\usepackage{hyperref}
%% \usepackage{authblk}
\usepackage[labelfont=bf]{caption} % make label for figure bold

% We will generate all images so they have a width \maxwidth. This means
% that they will get their normal width if they fit onto the page, but
% are scaled down if they would overflow the MAPgins.
\makeatletter
\def\maxwidth{\ifdim\Gin@nat@width>\linewidth\linewidth\else\Gin@nat@width\fi}
\def\maxheight{\ifdim\Gin@nat@height>\textheight\textheight\else\Gin@nat@height\fi}
\makeatother
\setkeys{Gin}{width=\maxwidth,height=\maxheight,keepaspectratio}

\title{Supplementary Materials 2. Modification of \plant\ model to
  include vapour pressure deficit with a coupled Farquhar-stomatal
  conductance model}

\author{Georges Kunstler, Daniel S. Falster, Richard G. FitzJohn}
\date{}
\affiliation{INRAE LESSEM, Grenoble, France and Department of Biological Sciences, Macquarie University,
  Sydney, Australia}
\date{}
\runninghead{}
\keywords{}

\usepackage{color}

\newcommand{\ud}{\ensuremath{\mathrm{d}}}
\newcommand{\sign}{\mathop{\mathrm{sign}}\nolimits}
\newcommand{\Rstar}{\ensuremath{R^*}}
\newcommand{\plant}{{\tt plant}}
\newcommand{\hmat}{\ensuremath{h_{\text{mat}}}}
\newcommand{\TODO}{{\color{red}\sc todo}}


\begin{document}

\mstitleshort
%% \mstitlepage
\parindent=1.5em
\addtolength{\parskip}{.3em}

% \begin{abstract}
% Abstract goes here\ldots
% \end{abstract}

\section{Introduction}

This supplementary materials presents the extension of the \plant\
model. First the photosynthesis model is modified to use a coupled Farquhar
photosynthesis - stomatal conductance model that include an effect of vapour pressure deficit
(vpd). Then we present the choice of the parameters of the model and the link with
$N_{area}$. Finally,  we include an effect of $N_{area}$ on the leaf
turnover rate vs
$LMA$ tradeoff.

\section{Development of the Farquhar photosynthesis model with vpd and Leaf N effects}

\subsection{Farquhar photosynthesis model  in \plant\ .}

The Farquhar model describes leaf photosynthesis as the minimum
of three potential rates $J_E$ the light limited rate, $J_C$ the CO2
rubisco limited rate, and $J_S$ the starch export limitation rate. The
last limitation by starch export (or triose-phosphate utilization)
limitation is generally neglected as it come into play only at very
high level of $CO_2$.

\begin{equation}
\label{eq:An}
A_n= min(J_E, J_C, J_s) - R_d.
\end{equation}

%% There is a lot of variant along this model but
%% overall they would all predict that $N_{area}$ would influence
%% $A_{max}$ by influencing either $J_C$ which is a function of $V_{cmax}$
%% which is related to $N_{area}$ or $J_E$ which is related to $J_{max}$
%% which related to $N_{area}$ as well. It is probably better to use
%% hyperbolic minimum rather than minimum function to avoid
%% discontinuity.

%% The key mechanisms which support that $N_{area}$ might allow to
%% adapt to low water availability is that increasing $N_{area}$ at a
%% given stomatal conductance $g_s$
%% increase $V_{cmax}$ and thus decrease the ratio $c_i/c_a$.


%% \begin{equation}
%% \label{eq:JC}
%% J_E = Q_p \frac{a \alpha (p_i - \Gamma_*)}{p_i + 2 \Gamma_*}.
%% \end{equation}

%% , and $Q_p$ is the incident flux of photo-synthetically active photon.

\begin{equation}
\label{eq:JC}
J_C= \frac{V_{cmax} (p_i - \Gamma_*)}{p_i + K_c (1+[o_2]/K_o)}.
\end{equation}

where $K_c$ and $K_o$ are the Michaelis constant for $CO_2$ and the competitive inhibition constant for $O_2$, and $V_{cmax}$ is the maximum catalytic capacity of Rubisco per leaf area.
$p_i$ is given by $p_i = P c_i$. And $\Gamma_* = \frac{[O_2]}{2\tau}$ ($\tau$ is a ratio of kinetic parameters describing the partitioning of RuNP to the carboxylase or oxygenase)

\begin{equation}
\label{eq:JCb}
J_E = J \frac{ c_i - \Gamma_*}{4c_i + 8 \Gamma_*}.
\end{equation}

where

\begin{equation}
\label{eq:Jlight}
J = \frac{ \alpha I + J_{max} - \sqrt{(\alpha I + J_{max})^2 - 4 \theta \alpha I J_{max}}}{2\theta}.
\end{equation}

with $\alpha$ the initial quantum yield (in mol of electron per mol of photon), $\theta$ the curvature of the light response and $J_{max}$ the maximum electron transfer,
see for instance \citet{Bernacchi-2009}.


\begin{equation}
\label{eq:JS}
J_S= V_{cmax}/2.
\end{equation}

To avoid discontinuity due to the minimum function between three
limiting rates, the minimum function is generally replaced by
hyperbolic minimums based on the following quadratics functions:

\begin{equation}
\label{eq:Q1}
\theta J_p^2 - J_p(J_E+J_C) + J_E J_C= 0.
\end{equation}

where $J_P$ is the min of $J_C$ and $J_E$.

\begin{equation}
\label{eq:Q2}
\beta A^2 - A(J_P+J_S) + J_P J_S= 0.
\end{equation}

where $A$ is the gross photosynthesis. $\theta$ and $\beta$ are constant describing the transition
between limitation (classically $\theta =1 - 1E-04$ and $\beta = 1 -
1E-07$). We used these hyperbolic minimum.


The kinetic parameters $K_c$, $K_o$, $V_{cmax}$, and $\tau$ change with temperature according to a $Q_{10}$ function ($k = k_{25} Q_{10}^{(T-25)/10}$).

We used the default parameters of \citep{Duursma-2015}.

\clearpage

\subsection{Link between $J_{max}$ \textit{vs.} $V_{cmax}$}

The hypothesis of a coordination between Calvin-Benson cycle limited
rate of assimilation and electron transport limited rate of
assimilation, lead to the assumption of a correlation between $V_{cmax}$ and
$J_{max}$. Different model to link them have been proposed in the
litterature.

\begin{itemize}

\item $J_{max} = 1.67 \times  V_{cmax}$ in \citet{Medlyn-2002}.
  
\item A power function of $V_{cmax}$ $J_{max} = exp(1.01) \times
  V_{cmax}^{0.89}$ in \citet{Walker-2014}.

\item  $J_{max} = exp(1.669) \times V_{cmax}^{0.75}$ in \citet{Kattge-2011} based on the TRY data.

\item $J_{max} = exp(1.425) \times V_{cmax}^{0.837}$ in \citet{Wullschleger-1993}.

\end{itemize}

The Figure \ref{fig:Vcmax_Jmax} show the different parametrisation.

\begin{figure}[ht]
\centering
\includegraphics{../figures/Param_FvC_Vcmax_Jmax.pdf}
\caption{\textbf{Different parametrisation of the link between $J_{max}$ and $V_{cmax}$.}
\label{fig:Vcmax_Jmax}}
\end{figure}

The slope of $J_{max}$ vs $V_{cmax}$ (in log log) affect the light level at which photosynthesis become RuBisCO limited. A shallower slope shift the light level to higher light value and a steeper slope to a lower light value. This will thus affect shade tolerance.

The link between $J_{max}$ and $V_{cmax}$ might however be more
complexe and depend on the light level and leaf P. According to
\citet{Walker-2014} the effect of SLA and leaf P are however
relatively small. So this is ok to use only $J_{max}$ in function of
$V_{cmax}$. We propose to use the classical parameterisation of \citet{Medlyn-2002} .

\subsection{Coupled photosynthesis and stomatal conductance models}

To include the effect of aridity we need to couple the photosynthesis model with a stomatal conductance model. In most approaches the stomatal conductance model is linked with Farquhar model with the Fick law:

\begin{equation}
\label{eq:fick}
A_n = g_s (c_a - c_i) = g_s c_a (1-\chi).
\end{equation}

Where $c_a$ and $c_i$ are respectively the $CO_2$ mole fraction of the air and in the leaf.

For the stomatal model several different models have been proposed
that range from empirical models to optimisation models and from
models that include or not soil water stress. If soil water stress is
include some approach also include competition for soil water between
trees \citep{Farrior-2013}. Here we will explore a relatively simple
and classical implementation with no effect of soil water stress and
no representation of water competition.

Below are presented the three most classical stomatal models.

The first model is the one of \citet{Collatz-1991} which connect the Farquhar photosynthesis model with the Ball Berry model which predict $g_s$. This is an empirical relationship based on the following equation:

\begin{equation}
\label{eq:gs-Ball}
g_s= g_0 + g_1 \frac{A_n h_s}{c_a}.
\end{equation}

where $h_s$ relative humidity at the leaf surface, and $c_a$ is the $CO_2$ mole fraction of the air (at leaf surface).

This has been modified by \citet{Leuning-1995} as the following empirical equation:

\begin{equation}
\label{eq:gs-Leuning}
g_s= g_0 + g_1 \frac{A_n }{(c_a - \Gamma)(1+D/D_0)}.
\end{equation}

Where $D$ is the air vapour pressure deficit (vpd).

Then several approach have been developed to predict stomatal conductance based on the optimisation of the summed cost of water loss and carbon gain \citep{Medlyn-2002,Prentice-2014,Wolf-2016,Sperry-2017}. One of the most used approach is the one developed by \citet{Medlyn-2011} as :

\begin{equation}
\label{eq:gs-Medlyn}
g_s \approx g_0 + (1 + \frac{g_1}{\sqrt{D}}) \frac{A_n }{c_a}.
\end{equation}


% \textit{NOT TO KEEP Then other authors have proposed to account for the impact of xylem cavitation \citet{Wolf-2016,Sperry-2016,Sperry-2017} in the model and also to maximise carbon gain rather than the water use efficiency \citet{Wolf-2016}. I think that \citep{Sperry-2016} propose a framework that would be easy to implement and to connect to a water budget in \plant\ . But as an initial step we could use either the model of \citet{Medlyn-2011} or \citet{Leuning-1995}. We need to keep in mind that these models that are based only on vpd are supposed to be applicable only to plant without soil water stress.}

The figure \ref{fig:photo_stomat} shows three different classical coupled models.

% \begin{figure}[ht]
% \centering
% \includegraphics{../figures/FvC_stomatal_BallBerry.pdf}
% \caption{\textbf{Coupled FvC and Ball Berry models with effect of $V_{cmax}$ and VPD.}
% \label{fig:photo_stomat_BB}}
% \end{figure}

% \begin{figure}[ht]
% \centering
% \includegraphics{../figures/FvC_stomatal_BBLeuning.pdf}
% \caption{\textbf{Coupled FvC and Leuning models with effect of $V_{cmax}$ and VPD.}
% \label{fig:photo_stomat_Leuning}}
% \end{figure}

% \begin{figure}[ht]
% \centering
% \includegraphics{../figures/FvC_stomatal_BBOpti.pdf}
% \caption{\textbf{Coupled FvC and Medlyn models with effect of $V_{cmax}$ and VPD.}
% \label{fig:photo_stomat_opti}}
% \end{figure}

\begin{figure}[ht]
\centering
\includegraphics{../figures/FvCstomatal.pdf}
\caption{\textbf{Coupled Farquhar and stomatal models with effect of
    $V_{cmax}$ and VPD for three stomatal conductance models, Ball
    Berry model, Leuning model, and Medlyn optimal model.}
\label{fig:photo_stomat}}
\end{figure}

\clearpage

The Ball Berry model lead to a very low photosynthesis at high vpd,
and has a discontinuity wich might be difficult to handel in numerical
simulation. The Medlyn optimal model lead also to a
discontinuity and has a relatively weak vpd effect. But more
importantly the Medlyn model is based on an optimisation approach that
ignore light competition whereas we do not want to be limited to an
optimisation approach. We thus decided to use the Leuning empirical
model (as in ED2). 

% The variation of the Leuning model with vpd seems to give rise to an annual model that is well approximated by the non-rectangular hyperbola (see Figure \ref{fig:photo_annu_fvc_vpd3}).

% \begin{figure}[ht]
% \centering
% \includegraphics{../figures/Annual_Photo_FvC_vpd3.pdf}
% \caption{\textbf{Annual photosynthesis model of FvC with the parametrisation of Troll, $\alpha = 0.3$, $\theta = 0.7$, $V_{cmax} = 38$ and $J_{max} = 1.67 * V_{cmax}$ coupled with the Leuning stomatal model and integrated over one year in the left panel.}
% \label{fig:photo_annu_fvc_vpd3}}
% \end{figure}

% So my proposition would be to run simulation with this coupled FvC-Leuning model along a gradient of vpd where the $V_{cmax}$ and $J_{max}$ are related to $N_{area}$ using the link proposed by \citep{Sakschewski-2015}.

\pagebreak


\subsection{Parametrisation of the Farquhar model}

The optimisation approach of $N_{area}$ in function drought assume a
link between $V_{cmax}$ and $N_{area}$ that need to be paramterised in
the photosynthesis model. Then, the light reponse of photosynthesis model needs to be parameterised.

\subsubsection{Parametrisation of $V_{cmax}$ \textit{vs.} $N_{area}$}

Photosynthetic rate should scale positively with leaf $N_{area}$ because of the large amount of nitrogen invested in RuBisCO. Several relationship have been proposed to link $V_{cmax}$ and $N_{area}$.

\begin{itemize}

\item \citet{Kattge-2009} proposed the relationship $V_{cmax} = a + b
  \times N_{area}$ with different value per biomes estimated either
  with $V_{cmax}$ only or also constrained by $A_{max}$ (unit g per m2 need to be converted in kg per m2).

\item \citet{Sakschewski-2015} proposed the relationship $V_{cmax} = 31.62 \times N_{area}^{0.801}$ (unit g per m2 need to be converted in kg per m2).

\item  \citet{Walker-2014} proposed that $ln(Vcmax) = 3.946 + 0.921
  \times log(N) + 0.121 \times log(P) + 0.281 \times log(N) \times
  log(P)$ and explored other model that include also LMA. To get a
  function of only N this should be $V_{cmax} = exp(3.712) \times
  N^{0.650}$ see  \citet{Walker-2017} (unit g per m2 need to be converted in kg per m2).

\item \citet{Domingues-2010} proposed several model based on N and P.
  If we look only at the one based only on N this is: $V_{cmax} =
  10^{1.57} \times N_{area}^{0.55}$ (Appendix C Table A2) (unit g per m2 need to be converted in kg per m2).

\end{itemize}

The Figure \ref{fig:Narea_Vcmax} shows the different parametrisation.

\begin{figure}[ht]
\centering
\includegraphics{../figures/Param_FvC_Narea_Vcmax.pdf}
\caption{\textbf{Different parametrization of the link between $N_{area}$ and $V_{cmax}$. The panel on the right present the different biomes estimate of \citet{Kattge-2011}.}
\label{fig:Narea_Vcmax}}
\end{figure}

Other traits such as leaf P and $LMA$ could also influence $V_{cmax}$
\citep{Walker-2014,Domingues-2010}. $LMA$ has little effect on the
link between $V_{cmax}$ and $N_{area}$.  Leaf P has little direct
effect on $V_{cmax}$ but an important interaction with $N_{area}$: at
low leaf P the slope of $V_{cmax}$ \textit{vs} $N_{area}$ is shallow
but at high leaf P this slope is steeper \citep{Walker-2014}.
% It is also possible to use an other approach where both $V_{cmax}$ and $J_{max}$
% are function of N and P rather than having Jmax a function of Vcmax (Domingues et al. 2010, Walker et al. 2014).
The objective is not to focus on P effect, so we propose to use the
most recent parameterisation using the TRY data of \citet{Sakschewski-2015}.

\clearpage

\subsubsection{Light response parametrisation of the Farquhar model}

We propose to choose the parameters of light response of the Farqhuar model
classicaly used for evergreen forest trees. The parameterisation has
important impact on the model and numerous type of parameterisation
exist \citep{Rogers-2017}. Here is a short list of existing parametrisation for the
light response part:
\begin{itemize}

\item In plantecophys \citep{Duursma-2015}  $\alpha = 0.24$ and $\theta = 0.85$

\item In \citet{Medlyn-2002} $\alpha = 0.3$ and $\theta = 0.9$

\item In TROLL \citet{Marechaux-2017} $\alpha = 0.075*4 = 0.3$ and $\theta = 0.7$ based on von Caemmerer 2000

\end{itemize}


\begin{figure}[ht]
\centering
\includegraphics{../figures/Photo_FvC_narea_vpd0.pdf}
\caption{\textbf{Photosynthesis model of \plant\ and Farquhar model
    for different parameterisation of the reponse curve at mean $N_{area}$ and a $vpd
    = 0$.}
\label{fig:photo0}}
\end{figure}

\begin{figure}[ht]
\centering
\includegraphics{../figures/Photo_FvC_narea_vpd1_5.pdf}
\caption{\textbf{Photosynthesis model of \plant\ and Farquhar model for different parameterisation of the reponse curve at mean $N_{area}$ and a $vpd = -1.5$.}
\label{fig:photo15}}
\end{figure}

\begin{figure}[ht]
\centering
\includegraphics{../figures/Photo_FvC_narea_vpd3.pdf}
\caption{\textbf{Photosynthesis model of \plant\ and Farquhar model for different parameterisation of the reponse curve at mean $N_{area}$ and a $vpd = -3$.}
\label{fig:photo3}}
\end{figure}

The figures \ref{fig:photo0} to \ref{fig:photo3} show the light response curves for
different parameterisation at vpd from 0 to -3. We will use the
parametrisation of \citet{Marechaux-2017}.

\clearpage

\subsubsection{Annual photosynthesis model}

In \plant\ the short term photosynthesis is used to derive an annual
photosynthesis model because the simulation approach is done at an
annual time step. \plant\ use a Michaelis
Menten function to approximate the annual photosynthesis rate. The Michaelis Menten approximation
is not working well for the Farquhar model, see Figures
\ref{fig:photo_annu_fvc0} to \ref{fig:photo_annu_fvc3}. We tested the exponential
function \citep{Chen-2016} ($A = p_1 *(1-e^{-p_2* E/p_1})$) and the
non-rectangular hyperbola, see figure
\ref{fig:photo_annu_fvc15}. We will use a non-rectangular hyperbola
which work well even at low vpd.

\begin{figure}[ht]
\centering
\includegraphics{../figures/Annual_Photo_FvC_narea_vpd0.pdf}
\caption{\textbf{Annual photosynthesis model of Farquhar model with
    the parametrisation of Troll, $\alpha = 0.3$, $\theta = 0.7$,
    $V_{cmax} = 38$ and $J_{max} = 1.67 * V_{cmax}$ for $ vpd = 0$.}
\label{fig:photo_annu_fvc0}}
\end{figure}

\begin{figure}[ht]
\centering
\includegraphics{../figures/Annual_Photo_FvC_narea_vpd1_5.pdf}
\caption{\textbf{Annual photosynthesis model of Farquhar model with
    the parametrisation of Troll, $\alpha = 0.3$, $\theta = 0.7$,
    $V_{cmax} = 38$ and $J_{max} = 1.67 * V_{cmax}$ for $vpd = -1.5$.}
\label{fig:photo_annu_fvc15}}
\end{figure}

\begin{figure}[ht]
\centering
\includegraphics{../figures/Annual_Photo_FvC_narea_vpd3.pdf}
\caption{\textbf{Annual photosynthesis model of Farquhar model with
    the parametrisation of Troll, $\alpha = 0.3$, $\theta = 0.7$,
    $V_{cmax} = 38$ and $J_{max} = 1.67 * V_{cmax}$ for $vpd = -3$.}
\label{fig:photo_annu_fvc3}}
\end{figure}


\clearpage

\subsection{Range of varriation of $vpd$ at constant temperature}

On the short term, $vpd$ and air temperature covary strongly. $vpd$
covaries with temperature according to its definition, but the correlation is
not perfect. To evaluate on which range we could vary $vpd$ at a
constant temperature we evaluated the link between $vpd$ and
temperature ($T$) at the global scale using the equations of \citep{Allen-1998} to compute the $vpd$ from
$T_{min}$, $T_{max}$, and relative humidity $RH$:

$$e_s(T) = 0.6108 * exp(17.27 * T / (T + 237.3))$$

$$e_s = (e_s(T_{min})+e_s(T_{max}))/2 $$

$$e_a = RH / 100 * e_s $$

$$vpd = e_a - e_s$$

$vpd$ is thus related to $T$ but also to $RH$. We use worldclim data
to derive annual $vpd$ in function $T_{min}$, $T_{max}$, and $RH$
globally (we thus ignore seasonal variation). At
global scale the link between $vpd$ and $T$ is given in the Figure \ref{fig:vpd_t}.
\begin{figure}[ht]
\centering
\includegraphics{../figures/vpd_t_mean.png}
\caption{\textbf{Link between vpd and T at global scale.}
\label{fig:vpd_t}}
\end{figure}

 For a temprature of 25C it is possible to vary vpd between -0.25 and
 -3 along the green line. We will explore this range of vpd at a
 constant temperature of 25C.
 
\clearpage
 
 \subsection{$N_{area}$ effect of leaf turnover rate}

The elevation of $LMA$ \textit{vs.} leaf turnover rate($LTR$) clearly vary with $N_{area}$ see Figure \ref{fig:narea_tradeoff}. 

\begin{figure}[ht]
\centering
\includegraphics{../figures/data_lma_ll_trade_off_narea.pdf}
\caption{\textbf{Trade offs between $LMA$ and LTR at different value of
    N area based on glopnet \citep{Wright-2004}.}
\label{fig:narea_tradeoff}}
\end{figure}

The intercept of the $LMA$ \textit{vs.} $LTR$ is linearly related to
$N_{area}$ as shown on the figure \ref{fig:narea_intercept} and there
is no significant effect on the slope. We thus also implemented this
effect in \plant\ to test the effect of an
additional cost of high $N_{area}$ on community assembly. The relathionship is
parametrised with $N_{area} / \overline{N_{area}}$ to center the
variation around $\overline{N_{area}}$ as done in previous version of
\plant\ . 

\begin{figure}[ht]
\centering
\includegraphics{../figures/data_B_kl_narea.pdf}
\caption{\textbf{Link between $N_{area}$ and intercept ($B_{kl1}$) and
    slope ($B_{kl2}$) of the $LMA$ \textit{vs.} $LTR$ (using 14 bins of $N_{area}$ with equal number of observations per bin.}
\label{fig:narea_intercept}}
\end{figure}


\clearpage

% \subsection{TODO SHORT-TERM}

% In the short-term it would be interesting to think about:

% \begin{itemize}

% \item predict vpd and a full water budget for the observed data (glopnet) to facilitate comparison with theoretical prediction (building on \citet{Prentice-2014}).

% \item re-run all simulation on cluster with new version of plant to fix last convergence problems.

% \end{itemize}


% \subsection{TODO LONG-TERM}

% In the long-term it would be interesting to think about:

% \begin{itemize}


% \item using stomatal model that is linked to the soil water potential (as in \citet{Sperry-2016} or
% \citet{Sterck-2011})

% \item look into FATES documentation to use their soil water stress
%   effect on photosyntehsis see https://fates-docs.readthedocs.io/en/latest/fates_tech_note.html#fundamental-photosynthetic-physiology-theory

% \item include a simple water budget to model tree competition for water

% \end{itemize}



%% The Fick's law close the system of equation:
%% \begin{equation}
%% \label{eq:fick}
%% g_s=\frac{A_d}{c_a - c_i}.
%% \end{equation}

%% $p_i = P c_i$

%% The models are combined by solving simultaneously for $c_i$ and $g_s$.

%% \citet{Scheiter-2009} use this photsynthesis model (with a min function).

%% Other version of the model consider that instead a of $Q_p$ the light incidence, $J_E$ is a function $J$ the electron transport rate which is it self a function of incident light (with non rectangular hyperbola) \citep{Sharkey-2007}.



%% \subparagraph{\citet{Haxeltine-1996}, photosynthesis model assuming optimal allocation of Leaf N and then all LPJ models}

%% In LPJ no $J_S$ and $V_{cmax}$ is optimised to maximise photosynthesis (thus optimum allocation of Leaf N). This optimum allocation results in equation 14 of \citet{Sitch-2008}.

%% Water stress is tracked by a water balaence model with a demand evapotranspiration $E_d$ and a supply determined $E_s$ by the water availability. Water stress occurs when $E_d > E_s$.

%% Then if non water stress conductance is given by the photosynthesis equation by:

%% \begin{equation}
%% \label{eq:fickLPJ}
%% g_s=g_{min} + \frac{1.6A_d}{c_a - c_i}.
%% \end{equation}

%% and for C3 plant the ratio $c_i / c_a$ is set to 0.7.

%% The equation \ref{eq:fickLPJ} provides non water stressed $g_s$ which then gives $E_{d}$ by

%% \begin{equation}
%% \label{eq:12Sitch}
%% E=E_p \alpha_m [1 - exp(\frac{- g_s}{g_m})].
%% \end{equation}

%% If the condition are water stressed ($E_{supply} < E_{demand}$ value
%% of $g_s$ is given by the equation \ref{eq:12Sitch} (assuming $E$ is
%% equal to $_{supply}$. Then $g_s$ can be used to solve Fick law and the
%% photosynthesis model.

%% \subparagraph{\citet{Sakschewski-2015}, add Leaf N in LPG photosynthesis deacreasing below optimal Leaf N}

%% $V_{cmax}$ is a power function of Leaf N (a function of SLA , even if
%% not the case in Wright et al. 2004). The photosynthesis model of LPJ
%% is based on optimum allocation to Leaf N (so optimal $V_{cmax opt}$) so
%% the actual $V_{cmax}$ is the munimum between the optimum and the one
%% predicted by leaf N. This is some how strange as the effect of leaf N
%% on $V_{cmax}$ is limited by the optimal value.

%% \subparagraph{\citet{Prentice-2014}, optimal $c_i/c_a$ for $V_{cmax}$ vs $g_s$}

%% Predict optimal $c_i/c_a$ to optimise $V_{cmax}$ and $G_s$ (evaporation) for different $D$ vapour pressure deficit.

%% \subparagraph{\citet{Johnson-1984}, model and its use in more recent model}

%% In this alternative model of photosynthesis, only the response to
%% light is considered (and this match the euqation of the light effect
%% on $J$). This is the model used in 'Plant'.

%% \begin{equation}
%% \label{eq:Johnson}
%% A_{N,PAR} = \frac{ \alpha I + A_{max} - \sqrt{(\alpha I + A_{max})^2 - 4 \theta \alpha I A_{max}}}{2\theta}.
%% \end{equation}

%% \subparagraph{\citet{Farrior-2013}}

%% Use a simpler version of the light response curve (instead of using
%% \citet{Johnson-1984}) they use a simpled staturantion function
%% ($A_L = min(AL, A_{max})$). Then water demande is linearly related to
%% $A_L$ and water supply is related to soil water potential with a two
%% step function no supply if below a treshold, above this treshold
%% increase with soil water potential (see Fig. A2). Soil water content
%% is lenearly related to soil water potential.

%% TODO \citet{Zavala-2005} based on \citet{Leuning-1995} for stomatial
%% conductance (a new version of \citet{Collatz-1991}).

%% \subparagraph{Summary and what to do next}

%% Water availability affect photosynthesis in two way. During the period
%% of non water stress how photosynthesis work (the paper of
%% \citet{Wright-2003} deal with that. \citet{Medlyn-2011} also deal
%% with optimal conductance in soil water limited plant.  For
%% \citet{Prentice-2014} as well, ?, but they also argue that vpd
%% represent the long term water availability.

%% In period of non water stress all plant have non limited supply of
%% water from the soil but the vpd is different inducing different
%% transpiration at the same $g_s$. In that case leaf N area control the
%% water use efficiency. Then water availability control also
%% the duration of the year over which soil water is limiting. For a non water
%% stressed plant, soil water supply limit the photosynthesis by
%% constraining the stomatal conducatance to match the supply.



%% There is two different approaches. In the first one, a water budget model
%% predict when soil water is limiting the value of $g_s$ that match
%% evapotranspiration allowed by the supply. (PhD of Lohier differ in the
%% sens that transpiration is not computed as a function of PET).

%% In the second one there is a direct effect of $A$, $c_a$ and the
%% relative humidity on the $g_s$ but not effect of soil water stress.

%% What we need is to have a Photosynthesis model with light effect and
%% $A_{max}$ related to Leaf N as it is currently the case that would be
%% $A_L$ and then add an effectof water stress that would limit to
%% $A_W$. The $A = min(A_L , A_N)$ thus this is different than affecting
%% directly $A_{max}$ as at low light level $A$ will not be affected.

%% Then we need to have a link between Leaf N and the way plant tolerate
%% water stress


%% One approach is to say that there two A one for non water stressed
%% conditions $A_L$ and one for water stressed conditions $A_W$. The
%% water availability variable is $E_S$ evapotransipiration supply
%% (daily, monthly ???) per unit area of leaf. Assuming non water
%% stressed condition we can
%% compute $A_L = f(V_{cmax}, c_i, I)$ assuming $c_i/c_a = 0.7$ (not
%% perfect as Leaf N shouild affect that and thus when water stress
%% occurs). Based on $A_L$ compute $g_{s pot}$ and $E_S$ per unit area of
%% leaf. Then compute $\omega = min(1, E_S/E_D)$. If $\omega = 1$ then $A
%% = A_L$. If $\omega < 1$ then compute $g_s$ from $E_S$ and then for
%% $c_i$ giving the lowest $A$.



%% I don't think that is necessarely need to have $J_C$ $J_E$ (and a
%% proper light response). Weed need water stress reducing $g_s$ and thus $A$ in function of water stress AET/PET. If we derive this model this give use for each AET a prediction $A$ at different leaf N predicting $V_{cmax}$ as in \citet{Sakschewski-2015}.

%% How to simplify that ??? we can


%% Check correspondance to Plant photsynthesis model which is based on Johnson, I. R. and Thornley, J. H. M. 1984. A model of instantaneous and daily canopy photosynthesis. - Journal of Theoretical Biology 107: 531–545.
%%  and is a non-rectangular hyperbola response to light where we could assume that $A_{max}$ is related to $V_{cmax}$ and thus leaf N.

\clearpage



\bibliographystyle{amnat}
\bibliography{references}
\end{document}
